%% Generated by Sphinx.
\def\sphinxdocclass{report}
\documentclass[letterpaper,10pt,brazil]{sphinxmanual}
\ifdefined\pdfpxdimen
   \let\sphinxpxdimen\pdfpxdimen\else\newdimen\sphinxpxdimen
\fi \sphinxpxdimen=.75bp\relax
\ifdefined\pdfimageresolution
    \pdfimageresolution= \numexpr \dimexpr1in\relax/\sphinxpxdimen\relax
\fi
%% let collapsible pdf bookmarks panel have high depth per default
\PassOptionsToPackage{bookmarksdepth=5}{hyperref}

\PassOptionsToPackage{warn}{textcomp}
\usepackage[utf8]{inputenc}
\ifdefined\DeclareUnicodeCharacter
% support both utf8 and utf8x syntaxes
  \ifdefined\DeclareUnicodeCharacterAsOptional
    \def\sphinxDUC#1{\DeclareUnicodeCharacter{"#1}}
  \else
    \let\sphinxDUC\DeclareUnicodeCharacter
  \fi
  \sphinxDUC{00A0}{\nobreakspace}
  \sphinxDUC{2500}{\sphinxunichar{2500}}
  \sphinxDUC{2502}{\sphinxunichar{2502}}
  \sphinxDUC{2514}{\sphinxunichar{2514}}
  \sphinxDUC{251C}{\sphinxunichar{251C}}
  \sphinxDUC{2572}{\textbackslash}
\fi
\usepackage{cmap}
\usepackage[T1]{fontenc}
\usepackage{amsmath,amssymb,amstext}
\usepackage{babel}



\usepackage{tgtermes}
\usepackage{tgheros}
\renewcommand{\ttdefault}{txtt}



\usepackage[Sonny]{fncychap}
\ChNameVar{\Large\normalfont\sffamily}
\ChTitleVar{\Large\normalfont\sffamily}
\usepackage{sphinx}

\fvset{fontsize=auto}
\usepackage{geometry}


% Include hyperref last.
\usepackage{hyperref}
% Fix anchor placement for figures with captions.
\usepackage{hypcap}% it must be loaded after hyperref.
% Set up styles of URL: it should be placed after hyperref.
\urlstyle{same}

\addto\captionsbrazil{\renewcommand{\contentsname}{Conteúdo:}}

\usepackage{sphinxmessages}
\setcounter{tocdepth}{0}



\title{Estruturas de arquivos: estrutarq}
\date{04 abr. 2022}
\release{0.1}
\author{Jander Moreira}
\newcommand{\sphinxlogo}{\vbox{}}
\renewcommand{\releasename}{Release}
\makeindex
\begin{document}

\ifdefined\shorthandoff
  \ifnum\catcode`\=\string=\active\shorthandoff{=}\fi
  \ifnum\catcode`\"=\active\shorthandoff{"}\fi
\fi

\pagestyle{empty}
\sphinxmaketitle
\pagestyle{plain}
\sphinxtableofcontents
\pagestyle{normal}
\phantomsection\label{\detokenize{index::doc}}



\chapter{Introdução}
\label{\detokenize{index:introducao}}
\sphinxAtStartPar
O pacote \sphinxcode{\sphinxupquote{estrutarq}} foi moldado para dar suporte ao livro \sphinxtitleref{Estruturas de arquivos: uma abordagem prática}.

\sphinxAtStartPar
A implementação desenhada é simplificada. Em especial:
\begin{itemize}
\item {} 
\sphinxAtStartPar
o código é voltado à legibilidade e não ao desempenho

\item {} 
\sphinxAtStartPar
controle e recuperação de erros são mantidos no nível mínimo, restrito ao âmbito de controle de exceções

\item {} 
\sphinxAtStartPar
aspectos de acesso simultâneo aos dados são ignorados e, assim, não estão disponíveis mecanismos de exclusão mútua ou escalonamento de acesso

\end{itemize}


\section{Conteúdo geral}
\label{\detokenize{index:conteudo-geral}}
\sphinxstepscope


\subsection{Pacote \sphinxstyleliteralintitle{\sphinxupquote{estrutarq.dado}}}
\label{\detokenize{estrutarq.dado:pacote-estrutarq-dado}}\label{\detokenize{estrutarq.dado::doc}}

\subsubsection{Módulo \sphinxstyleliteralintitle{\sphinxupquote{estrutarq.dado\_comum}}}
\label{\detokenize{estrutarq.dado:module-estrutarq.dado.dado_comum}}\label{\detokenize{estrutarq.dado:modulo-estrutarq-dado-comum}}\index{módulo@\spxentry{módulo}!estrutarq.dado.dado\_comum@\spxentry{estrutarq.dado.dado\_comum}}\index{estrutarq.dado.dado\_comum@\spxentry{estrutarq.dado.dado\_comum}!módulo@\spxentry{módulo}}
\sphinxAtStartPar
Estruturação de dados para armazenamento interno, gravação e leitura, usando
representações diversas:
\begin{itemize}
\item {} 
\sphinxAtStartPar
Em representação bruta

\item {} 
\sphinxAtStartPar
Com terminador

\item {} 
\sphinxAtStartPar
Prefixada pelo comprimento

\item {} 
\sphinxAtStartPar
Em formato binário

\item {} 
\sphinxAtStartPar
De comprimento fixo predefinido

\end{itemize}

\sphinxAtStartPar
Licença: GNU GENERAL PUBLIC LICENSE V.3, 2007

\sphinxAtStartPar
Jander Moreira, 2022

\sphinxAtStartPar
As classes providas pelo módulo, importadas direramente de \sphinxcode{\sphinxupquote{estrutarq.dado}}, são a interface para a criação das organizações de dados disponíveis.


\paragraph{Dado bruto}
\label{\detokenize{estrutarq.dado:dado-bruto}}\index{DadoBruto (classe em estrutarq.dado)@\spxentry{DadoBruto}\spxextra{classe em estrutarq.dado}}

\begin{fulllineitems}
\phantomsection\label{\detokenize{estrutarq.dado:estrutarq.dado.DadoBruto}}
\pysigstartsignatures
\pysigline{\sphinxbfcode{\sphinxupquote{class\DUrole{w}{  }}}\sphinxcode{\sphinxupquote{estrutarq.dado.}}\sphinxbfcode{\sphinxupquote{DadoBruto}}}
\pysigstopsignatures
\sphinxAtStartPar
Base: {\hyperref[\detokenize{estrutarq.dado:estrutarq.dado.DadoBasico}]{\sphinxcrossref{\sphinxcode{\sphinxupquote{estrutarq.dado.dado\_comum.DadoBasico}}}}}

\sphinxAtStartPar
Classe para dado em forma bruta, ou seja, sem acréscimo de qualquer forma
de organização de dados.

\sphinxAtStartPar
Campos brutos não possuem aplicação prática e são usados apenas para fins
didáticos.
\index{adicione\_formatacao() (método estrutarq.dado.DadoBruto)@\spxentry{adicione\_formatacao()}\spxextra{método estrutarq.dado.DadoBruto}}

\begin{fulllineitems}
\phantomsection\label{\detokenize{estrutarq.dado:estrutarq.dado.DadoBruto.adicione_formatacao}}
\pysigstartsignatures
\pysiglinewithargsret{\sphinxbfcode{\sphinxupquote{adicione\_formatacao}}}{\emph{\DUrole{n}{dado}\DUrole{p}{:}\DUrole{w}{  }\DUrole{n}{\sphinxhref{https://docs.python.org/3/library/stdtypes.html\#bytes}{bytes}}}}{{ $\rightarrow$ \sphinxhref{https://docs.python.org/3/library/stdtypes.html\#bytes}{bytes}}}
\pysigstopsignatures
\sphinxAtStartPar
Para dado bruto não há acréscimo de bytes de organização de dados e
o dado é repassado sem modificação.
\begin{quote}\begin{description}
\item[{Parâmetros}] \leavevmode
\sphinxAtStartPar
\sphinxstyleliteralstrong{\sphinxupquote{dado}} (\sphinxhref{https://docs.python.org/3/library/stdtypes.html\#bytes}{\sphinxstyleliteralemphasis{\sphinxupquote{bytes}}}) \textendash{} bytes do dado

\item[{Retorna}] \leavevmode
\sphinxAtStartPar
bytes do dado inalterados

\item[{Tipo de retorno}] \leavevmode
\sphinxAtStartPar
\sphinxhref{https://docs.python.org/3/library/stdtypes.html\#bytes}{bytes}

\end{description}\end{quote}

\end{fulllineitems}

\index{leia\_de\_arquivo() (método estrutarq.dado.DadoBruto)@\spxentry{leia\_de\_arquivo()}\spxextra{método estrutarq.dado.DadoBruto}}

\begin{fulllineitems}
\phantomsection\label{\detokenize{estrutarq.dado:estrutarq.dado.DadoBruto.leia_de_arquivo}}
\pysigstartsignatures
\pysiglinewithargsret{\sphinxbfcode{\sphinxupquote{leia\_de\_arquivo}}}{\emph{\DUrole{n}{arquivo}\DUrole{p}{:}\DUrole{w}{  }\DUrole{n}{\sphinxhref{https://docs.python.org/3/library/typing.html\#typing.BinaryIO}{BinaryIO}}}}{}
\pysigstopsignatures
\sphinxAtStartPar
Recuperação de um dado lido de um arquivo (inviável para
dado bruto).
\begin{quote}\begin{description}
\item[{Parâmetros}] \leavevmode
\sphinxAtStartPar
\sphinxstyleliteralstrong{\sphinxupquote{arquivo}} (\sphinxstyleliteralemphasis{\sphinxupquote{BinaryIO}}) \textendash{} arquivo binário aberto com permissão de leitura

\item[{Levanta}] \leavevmode
\sphinxAtStartPar
\sphinxstyleliteralstrong{\sphinxupquote{NotImplemented}} \textendash{} se o método for acidentalmente chamado

\end{description}\end{quote}

\end{fulllineitems}

\index{leia\_de\_bytes() (método estrutarq.dado.DadoBruto)@\spxentry{leia\_de\_bytes()}\spxextra{método estrutarq.dado.DadoBruto}}

\begin{fulllineitems}
\phantomsection\label{\detokenize{estrutarq.dado:estrutarq.dado.DadoBruto.leia_de_bytes}}
\pysigstartsignatures
\pysiglinewithargsret{\sphinxbfcode{\sphinxupquote{leia\_de\_bytes}}}{\emph{\DUrole{n}{sequencia}\DUrole{p}{:}\DUrole{w}{  }\DUrole{n}{\sphinxhref{https://docs.python.org/3/library/stdtypes.html\#bytes}{bytes}}}}{}
\pysigstopsignatures
\sphinxAtStartPar
Recuperação de um dado extraído de uma sequência de bytes (inviável
para dado bruto).
\begin{quote}\begin{description}
\item[{Parâmetros}] \leavevmode
\sphinxAtStartPar
\sphinxstyleliteralstrong{\sphinxupquote{sequencia}} (\sphinxhref{https://docs.python.org/3/library/stdtypes.html\#bytes}{\sphinxstyleliteralemphasis{\sphinxupquote{bytes}}}) \textendash{} sequência de bytes

\item[{Levanta}] \leavevmode
\sphinxAtStartPar
\sphinxstyleliteralstrong{\sphinxupquote{NotImplemented}} \textendash{} se o método for acidentalmente chamado

\end{description}\end{quote}

\end{fulllineitems}

\index{remova\_formatacao() (método estrutarq.dado.DadoBruto)@\spxentry{remova\_formatacao()}\spxextra{método estrutarq.dado.DadoBruto}}

\begin{fulllineitems}
\phantomsection\label{\detokenize{estrutarq.dado:estrutarq.dado.DadoBruto.remova_formatacao}}
\pysigstartsignatures
\pysiglinewithargsret{\sphinxbfcode{\sphinxupquote{remova\_formatacao}}}{\emph{\DUrole{n}{sequencia}\DUrole{p}{:}\DUrole{w}{  }\DUrole{n}{\sphinxhref{https://docs.python.org/3/library/stdtypes.html\#bytes}{bytes}}}}{{ $\rightarrow$ \sphinxhref{https://docs.python.org/3/library/stdtypes.html\#bytes}{bytes}}}
\pysigstopsignatures
\sphinxAtStartPar
Para o dado bruto não há bytes de organização a sequência de bytes é
repassada sem modificação.
\begin{quote}\begin{description}
\item[{Parâmetros}] \leavevmode
\sphinxAtStartPar
\sphinxstyleliteralstrong{\sphinxupquote{sequencia}} (\sphinxhref{https://docs.python.org/3/library/stdtypes.html\#bytes}{\sphinxstyleliteralemphasis{\sphinxupquote{bytes}}}) \textendash{} uma sequência de bytes

\item[{Retorna}] \leavevmode
\sphinxAtStartPar
a sequência inalterada

\item[{Tipo de retorno}] \leavevmode
\sphinxAtStartPar
\sphinxhref{https://docs.python.org/3/library/stdtypes.html\#bytes}{bytes}

\end{description}\end{quote}

\end{fulllineitems}


\end{fulllineitems}



\paragraph{Dado com terminador}
\label{\detokenize{estrutarq.dado:dado-com-terminador}}\index{DadoTerminador (classe em estrutarq.dado)@\spxentry{DadoTerminador}\spxextra{classe em estrutarq.dado}}

\begin{fulllineitems}
\phantomsection\label{\detokenize{estrutarq.dado:estrutarq.dado.DadoTerminador}}
\pysigstartsignatures
\pysiglinewithargsret{\sphinxbfcode{\sphinxupquote{class\DUrole{w}{  }}}\sphinxcode{\sphinxupquote{estrutarq.dado.}}\sphinxbfcode{\sphinxupquote{DadoTerminador}}}{\emph{\DUrole{n}{terminador}\DUrole{p}{:}\DUrole{w}{  }\DUrole{n}{\sphinxhref{https://docs.python.org/3/library/stdtypes.html\#bytes}{bytes}}}}{}
\pysigstopsignatures
\sphinxAtStartPar
Base: {\hyperref[\detokenize{estrutarq.dado:estrutarq.dado.DadoBasico}]{\sphinxcrossref{\sphinxcode{\sphinxupquote{estrutarq.dado.dado\_comum.DadoBasico}}}}}

\sphinxAtStartPar
Classe para implementação de dados com terminador. O dado é tratado como
uma sequência de bytes à qual um byte predefinido
({\hyperref[\detokenize{estrutarq.dado:estrutarq.dado.DadoTerminador.terminador}]{\sphinxcrossref{\sphinxcode{\sphinxupquote{terminador}}}}}) é
acrescentado ao final para demarcar o fim dos dados. A existência do
valor do byte terminador na sequência de dados é tratada com a técnica
de enchimento de bytes (implementada em
{\hyperref[\detokenize{estrutarq.dado:estrutarq.dado.DadoBasico}]{\sphinxcrossref{\sphinxcode{\sphinxupquote{DadoBasico}}}}}).
\begin{quote}\begin{description}
\item[{Parâmetros}] \leavevmode
\sphinxAtStartPar
\sphinxstyleliteralstrong{\sphinxupquote{terminador}} (\sphinxhref{https://docs.python.org/3/library/stdtypes.html\#bytes}{\sphinxstyleliteralemphasis{\sphinxupquote{bytes}}}) \textendash{} um byte a ser usado como terminador

\end{description}\end{quote}
\index{adicione\_formatacao() (método estrutarq.dado.DadoTerminador)@\spxentry{adicione\_formatacao()}\spxextra{método estrutarq.dado.DadoTerminador}}

\begin{fulllineitems}
\phantomsection\label{\detokenize{estrutarq.dado:estrutarq.dado.DadoTerminador.adicione_formatacao}}
\pysigstartsignatures
\pysiglinewithargsret{\sphinxbfcode{\sphinxupquote{adicione\_formatacao}}}{\emph{\DUrole{n}{dado}\DUrole{p}{:}\DUrole{w}{  }\DUrole{n}{\sphinxhref{https://docs.python.org/3/library/stdtypes.html\#bytes}{bytes}}}}{{ $\rightarrow$ \sphinxhref{https://docs.python.org/3/library/stdtypes.html\#bytes}{bytes}}}
\pysigstopsignatures
\sphinxAtStartPar
Formatação do dado: uso de ‘byte stuffing’ para permitir o byte
terminador como dado e acréscimo do byte terminador.
\begin{quote}\begin{description}
\item[{Parâmetros}] \leavevmode
\sphinxAtStartPar
\sphinxstyleliteralstrong{\sphinxupquote{dado}} (\sphinxhref{https://docs.python.org/3/library/stdtypes.html\#bytes}{\sphinxstyleliteralemphasis{\sphinxupquote{bytes}}}) \textendash{} sequência de bytes do dado

\item[{Retorna}] \leavevmode
\sphinxAtStartPar
a sequência de dados enchida e com o acréscimo do terminador
ao final

\item[{Tipo de retorno}] \leavevmode
\sphinxAtStartPar
\sphinxhref{https://docs.python.org/3/library/stdtypes.html\#bytes}{bytes}

\end{description}\end{quote}

\end{fulllineitems}

\index{leia\_de\_arquivo() (método estrutarq.dado.DadoTerminador)@\spxentry{leia\_de\_arquivo()}\spxextra{método estrutarq.dado.DadoTerminador}}

\begin{fulllineitems}
\phantomsection\label{\detokenize{estrutarq.dado:estrutarq.dado.DadoTerminador.leia_de_arquivo}}
\pysigstartsignatures
\pysiglinewithargsret{\sphinxbfcode{\sphinxupquote{leia\_de\_arquivo}}}{\emph{\DUrole{n}{arquivo}\DUrole{p}{:}\DUrole{w}{  }\DUrole{n}{\sphinxhref{https://docs.python.org/3/library/typing.html\#typing.BinaryIO}{BinaryIO}}}}{{ $\rightarrow$ \sphinxhref{https://docs.python.org/3/library/stdtypes.html\#bytes}{bytes}}}
\pysigstopsignatures
\sphinxAtStartPar
Leitura de um único dado com terminador. A leitura é feita byte a
byte até que o byte terminador seja encontrado. Bytes terminadores
enchidos são restaurados, mas não determinam o fim da busca. O
enchimento de bytes é removido.
\begin{quote}\begin{description}
\item[{Parâmetros}] \leavevmode
\sphinxAtStartPar
\sphinxstyleliteralstrong{\sphinxupquote{arquivo}} \textendash{} arquivo binário aberto com permissão de leitura

\item[{Retorna}] \leavevmode
\sphinxAtStartPar
a sequência de bytes do dado sem o terminador

\item[{Levanta}] \leavevmode
\sphinxAtStartPar
\sphinxhref{https://docs.python.org/3/library/exceptions.html\#EOFError}{\sphinxstyleliteralstrong{\sphinxupquote{EOFError}}} \textendash{} se o fim do arquivo for atingido antes de o byte
terminador ser encontrado

\end{description}\end{quote}

\end{fulllineitems}

\index{leia\_de\_bytes() (método estrutarq.dado.DadoTerminador)@\spxentry{leia\_de\_bytes()}\spxextra{método estrutarq.dado.DadoTerminador}}

\begin{fulllineitems}
\phantomsection\label{\detokenize{estrutarq.dado:estrutarq.dado.DadoTerminador.leia_de_bytes}}
\pysigstartsignatures
\pysiglinewithargsret{\sphinxbfcode{\sphinxupquote{leia\_de\_bytes}}}{\emph{\DUrole{n}{sequencia}\DUrole{p}{:}\DUrole{w}{  }\DUrole{n}{\sphinxhref{https://docs.python.org/3/library/stdtypes.html\#bytes}{bytes}}}}{{ $\rightarrow$ \sphinxhref{https://docs.python.org/3/library/stdtypes.html\#tuple}{tuple}\DUrole{p}{{[}}\sphinxhref{https://docs.python.org/3/library/stdtypes.html\#bytes}{bytes}\DUrole{p}{,}\DUrole{w}{  }\sphinxhref{https://docs.python.org/3/library/stdtypes.html\#bytes}{bytes}\DUrole{p}{{]}}}}
\pysigstopsignatures
\sphinxAtStartPar
Recuperação de um dado individual de uma sequência de bytes,
retornando o dado até o terminador e o restante da sequência depois
do terminador.
\begin{quote}\begin{description}
\item[{Parâmetros}] \leavevmode
\sphinxAtStartPar
\sphinxstyleliteralstrong{\sphinxupquote{sequencia}} (\sphinxhref{https://docs.python.org/3/library/stdtypes.html\#bytes}{\sphinxstyleliteralemphasis{\sphinxupquote{bytes}}}) \textendash{} uma sequência de bytes

\item[{Retorna}] \leavevmode
\sphinxAtStartPar
uma tupla contendo os bytes dos dados e a sequência de bytes
restante, excluindo\sphinxhyphen{}se de ambas o terminador

\item[{Tipo de retorno}] \leavevmode
\sphinxAtStartPar
\sphinxhref{https://docs.python.org/3/library/stdtypes.html\#tuple}{tuple}{[}\sphinxhref{https://docs.python.org/3/library/stdtypes.html\#bytes}{bytes}, \sphinxhref{https://docs.python.org/3/library/stdtypes.html\#bytes}{bytes}{]}

\end{description}\end{quote}

\end{fulllineitems}

\index{remova\_formatacao() (método estrutarq.dado.DadoTerminador)@\spxentry{remova\_formatacao()}\spxextra{método estrutarq.dado.DadoTerminador}}

\begin{fulllineitems}
\phantomsection\label{\detokenize{estrutarq.dado:estrutarq.dado.DadoTerminador.remova_formatacao}}
\pysigstartsignatures
\pysiglinewithargsret{\sphinxbfcode{\sphinxupquote{remova\_formatacao}}}{\emph{\DUrole{n}{sequencia}\DUrole{p}{:}\DUrole{w}{  }\DUrole{n}{\sphinxhref{https://docs.python.org/3/library/stdtypes.html\#bytes}{bytes}}}}{{ $\rightarrow$ \sphinxhref{https://docs.python.org/3/library/stdtypes.html\#bytes}{bytes}}}
\pysigstopsignatures
\sphinxAtStartPar
Desformatação do dado: remoção dos bytes de enchimento e também do
byte terminador.
\begin{quote}\begin{description}
\item[{Parâmetros}] \leavevmode
\sphinxAtStartPar
\sphinxstyleliteralstrong{\sphinxupquote{sequencia}} \textendash{} sequência de bytes de dados

\item[{Retorna}] \leavevmode
\sphinxAtStartPar
sequência de bytes de dados, esvaziada e sem terminador

\item[{Tipo de retorno}] \leavevmode
\sphinxAtStartPar
\sphinxhref{https://docs.python.org/3/library/stdtypes.html\#bytes}{bytes}

\item[{Levanta}] \leavevmode
\sphinxAtStartPar
\sphinxhref{https://docs.python.org/3/library/exceptions.html\#TypeError}{\sphinxstyleliteralstrong{\sphinxupquote{TypeError}}} \textendash{} se o terminador não estiver presente na sequência
esvaziada

\end{description}\end{quote}

\end{fulllineitems}

\index{terminador (propriedade estrutarq.dado.DadoTerminador )@\spxentry{terminador}\spxextra{propriedade estrutarq.dado.DadoTerminador }}

\begin{fulllineitems}
\phantomsection\label{\detokenize{estrutarq.dado:estrutarq.dado.DadoTerminador.terminador}}
\pysigstartsignatures
\pysigline{\sphinxbfcode{\sphinxupquote{property\DUrole{w}{  }}}\sphinxbfcode{\sphinxupquote{terminador}}\sphinxbfcode{\sphinxupquote{\DUrole{p}{:}\DUrole{w}{  }\sphinxhref{https://docs.python.org/3/library/stdtypes.html\#bytes}{bytes}}}}
\pysigstopsignatures
\sphinxAtStartPar
Byte simples usado como terminador.

\end{fulllineitems}


\end{fulllineitems}



\paragraph{Dado prefixado pelo comprimento}
\label{\detokenize{estrutarq.dado:dado-prefixado-pelo-comprimento}}\index{DadoPrefixado (classe em estrutarq.dado)@\spxentry{DadoPrefixado}\spxextra{classe em estrutarq.dado}}

\begin{fulllineitems}
\phantomsection\label{\detokenize{estrutarq.dado:estrutarq.dado.DadoPrefixado}}
\pysigstartsignatures
\pysigline{\sphinxbfcode{\sphinxupquote{class\DUrole{w}{  }}}\sphinxcode{\sphinxupquote{estrutarq.dado.}}\sphinxbfcode{\sphinxupquote{DadoPrefixado}}}
\pysigstopsignatures
\sphinxAtStartPar
Base: {\hyperref[\detokenize{estrutarq.dado:estrutarq.dado.DadoBasico}]{\sphinxcrossref{\sphinxcode{\sphinxupquote{estrutarq.dado.dado\_comum.DadoBasico}}}}}

\sphinxAtStartPar
Classe dado prefixados pelo seu comprimento
\index{adicione\_formatacao() (método estrutarq.dado.DadoPrefixado)@\spxentry{adicione\_formatacao()}\spxextra{método estrutarq.dado.DadoPrefixado}}

\begin{fulllineitems}
\phantomsection\label{\detokenize{estrutarq.dado:estrutarq.dado.DadoPrefixado.adicione_formatacao}}
\pysigstartsignatures
\pysiglinewithargsret{\sphinxbfcode{\sphinxupquote{adicione\_formatacao}}}{\emph{\DUrole{n}{dado}\DUrole{p}{:}\DUrole{w}{  }\DUrole{n}{\sphinxhref{https://docs.python.org/3/library/stdtypes.html\#bytes}{bytes}}}}{{ $\rightarrow$ \sphinxhref{https://docs.python.org/3/library/stdtypes.html\#bytes}{bytes}}}
\pysigstopsignatures
\sphinxAtStartPar
Formatação do dado: acréscimo do prefixo binário com comprimento
(2 bytes, big\sphinxhyphen{}endian, sem sinal).
\begin{quote}\begin{description}
\item[{Parâmetros}] \leavevmode
\sphinxAtStartPar
\sphinxstyleliteralstrong{\sphinxupquote{dado}} \textendash{} valor do dado

\item[{Retorna}] \leavevmode
\sphinxAtStartPar
o dado formatado

\end{description}\end{quote}

\end{fulllineitems}

\index{leia\_de\_arquivo() (método estrutarq.dado.DadoPrefixado)@\spxentry{leia\_de\_arquivo()}\spxextra{método estrutarq.dado.DadoPrefixado}}

\begin{fulllineitems}
\phantomsection\label{\detokenize{estrutarq.dado:estrutarq.dado.DadoPrefixado.leia_de_arquivo}}
\pysigstartsignatures
\pysiglinewithargsret{\sphinxbfcode{\sphinxupquote{leia\_de\_arquivo}}}{\emph{\DUrole{n}{arquivo}\DUrole{p}{:}\DUrole{w}{  }\DUrole{n}{\sphinxhref{https://docs.python.org/3/library/typing.html\#typing.BinaryIO}{BinaryIO}}}}{{ $\rightarrow$ \sphinxhref{https://docs.python.org/3/library/stdtypes.html\#bytes}{bytes}}}
\pysigstopsignatures
\sphinxAtStartPar
Leitura de um único dado prefixado pelo comprimento.
\begin{quote}\begin{description}
\item[{Parâmetros}] \leavevmode
\sphinxAtStartPar
\sphinxstyleliteralstrong{\sphinxupquote{arquivo}} \textendash{} arquivo binário aberto com permissão de leitura

\item[{Retorna}] \leavevmode
\sphinxAtStartPar
os bytes do dado

\end{description}\end{quote}

\sphinxAtStartPar
O comprimento é armazenado como um inteiro de 2 bytes, big\sphinxhyphen{}endian.
Em caso de falha na leitura é lançada a exceção EOFError

\end{fulllineitems}

\index{leia\_de\_bytes() (método estrutarq.dado.DadoPrefixado)@\spxentry{leia\_de\_bytes()}\spxextra{método estrutarq.dado.DadoPrefixado}}

\begin{fulllineitems}
\phantomsection\label{\detokenize{estrutarq.dado:estrutarq.dado.DadoPrefixado.leia_de_bytes}}
\pysigstartsignatures
\pysiglinewithargsret{\sphinxbfcode{\sphinxupquote{leia\_de\_bytes}}}{\emph{\DUrole{n}{sequencia: bytes) \sphinxhyphen{}\textgreater{} (\textless{}class \textquotesingle{}bytes\textquotesingle{}\textgreater{}}}, \emph{\DUrole{n}{\textless{}class \textquotesingle{}bytes\textquotesingle{}\textgreater{}}}}{}
\pysigstopsignatures
\sphinxAtStartPar
Recuperação de um dado individual de uma sequência de bytes,
retornando o dado sem o prefixo e o restante da sequência.
\begin{quote}\begin{description}
\item[{Parâmetros}] \leavevmode
\sphinxAtStartPar
\sphinxstyleliteralstrong{\sphinxupquote{sequencia}} \textendash{} uma sequência de bytes

\item[{Retorna}] \leavevmode
\sphinxAtStartPar
os bytes de dados e o restante da sequência

\end{description}\end{quote}

\end{fulllineitems}

\index{remova\_formatacao() (método estrutarq.dado.DadoPrefixado)@\spxentry{remova\_formatacao()}\spxextra{método estrutarq.dado.DadoPrefixado}}

\begin{fulllineitems}
\phantomsection\label{\detokenize{estrutarq.dado:estrutarq.dado.DadoPrefixado.remova_formatacao}}
\pysigstartsignatures
\pysiglinewithargsret{\sphinxbfcode{\sphinxupquote{remova\_formatacao}}}{\emph{\DUrole{n}{sequencia}\DUrole{p}{:}\DUrole{w}{  }\DUrole{n}{\sphinxhref{https://docs.python.org/3/library/stdtypes.html\#bytes}{bytes}}}}{{ $\rightarrow$ \sphinxhref{https://docs.python.org/3/library/stdtypes.html\#bytes}{bytes}}}
\pysigstopsignatures
\sphinxAtStartPar
Desformatação do dado: remoção dos dois bytes do comprimento.
\begin{quote}\begin{description}
\item[{Parâmetros}] \leavevmode
\sphinxAtStartPar
\sphinxstyleliteralstrong{\sphinxupquote{sequencia}} \textendash{} bytes de dados

\item[{Retorna}] \leavevmode
\sphinxAtStartPar
dado efetivo, sem o prefixo de comprimento

\end{description}\end{quote}

\end{fulllineitems}


\end{fulllineitems}



\paragraph{Dado binário}
\label{\detokenize{estrutarq.dado:dado-binario}}\index{DadoBinario (classe em estrutarq.dado)@\spxentry{DadoBinario}\spxextra{classe em estrutarq.dado}}

\begin{fulllineitems}
\phantomsection\label{\detokenize{estrutarq.dado:estrutarq.dado.DadoBinario}}
\pysigstartsignatures
\pysiglinewithargsret{\sphinxbfcode{\sphinxupquote{class\DUrole{w}{  }}}\sphinxcode{\sphinxupquote{estrutarq.dado.}}\sphinxbfcode{\sphinxupquote{DadoBinario}}}{\emph{\DUrole{n}{comprimento}\DUrole{p}{:}\DUrole{w}{  }\DUrole{n}{\sphinxhref{https://docs.python.org/3/library/functions.html\#int}{int}}}}{}
\pysigstopsignatures
\sphinxAtStartPar
Base: {\hyperref[\detokenize{estrutarq.dado:estrutarq.dado.DadoBasico}]{\sphinxcrossref{\sphinxcode{\sphinxupquote{estrutarq.dado.dado\_comum.DadoBasico}}}}}

\sphinxAtStartPar
Classe para dados binários com um determinado comprimento em bytes.
O comprimento é fixo.
\begin{quote}\begin{description}
\item[{Parâmetros}] \leavevmode
\sphinxAtStartPar
\sphinxstyleliteralstrong{\sphinxupquote{comprimento}} (\sphinxhref{https://docs.python.org/3/library/functions.html\#int}{\sphinxstyleliteralemphasis{\sphinxupquote{int}}}) \textendash{} comprimento em bytes do valor a ser armazenado

\end{description}\end{quote}
\index{adicione\_formatacao() (método estrutarq.dado.DadoBinario)@\spxentry{adicione\_formatacao()}\spxextra{método estrutarq.dado.DadoBinario}}

\begin{fulllineitems}
\phantomsection\label{\detokenize{estrutarq.dado:estrutarq.dado.DadoBinario.adicione_formatacao}}
\pysigstartsignatures
\pysiglinewithargsret{\sphinxbfcode{\sphinxupquote{adicione\_formatacao}}}{\emph{\DUrole{n}{dado}\DUrole{p}{:}\DUrole{w}{  }\DUrole{n}{\sphinxhref{https://docs.python.org/3/library/stdtypes.html\#bytes}{bytes}}}}{{ $\rightarrow$ \sphinxhref{https://docs.python.org/3/library/stdtypes.html\#bytes}{bytes}}}
\pysigstopsignatures
\sphinxAtStartPar
Formatação do dado: apenas repassa o dado binário.
\begin{quote}\begin{description}
\item[{Parâmetros}] \leavevmode
\sphinxAtStartPar
\sphinxstyleliteralstrong{\sphinxupquote{dado}} \textendash{} valor binário

\item[{Retorna}] \leavevmode
\sphinxAtStartPar
o dado formatado

\end{description}\end{quote}

\end{fulllineitems}

\index{leia\_de\_arquivo() (método estrutarq.dado.DadoBinario)@\spxentry{leia\_de\_arquivo()}\spxextra{método estrutarq.dado.DadoBinario}}

\begin{fulllineitems}
\phantomsection\label{\detokenize{estrutarq.dado:estrutarq.dado.DadoBinario.leia_de_arquivo}}
\pysigstartsignatures
\pysiglinewithargsret{\sphinxbfcode{\sphinxupquote{leia\_de\_arquivo}}}{\emph{\DUrole{n}{arquivo}\DUrole{p}{:}\DUrole{w}{  }\DUrole{n}{\sphinxhref{https://docs.python.org/3/library/typing.html\#typing.BinaryIO}{BinaryIO}}}}{{ $\rightarrow$ \sphinxhref{https://docs.python.org/3/library/stdtypes.html\#bytes}{bytes}}}
\pysigstopsignatures
\sphinxAtStartPar
Recuperação dos bytes do valor binário a partir de um arquivo.
\begin{quote}\begin{description}
\item[{Parâmetros}] \leavevmode
\sphinxAtStartPar
\sphinxstyleliteralstrong{\sphinxupquote{arquivo}} \textendash{} arquivo binário aberto com permissão de leitura

\item[{Retorna}] \leavevmode
\sphinxAtStartPar
a sequência de bytes lidos

\end{description}\end{quote}

\end{fulllineitems}

\index{leia\_de\_bytes() (método estrutarq.dado.DadoBinario)@\spxentry{leia\_de\_bytes()}\spxextra{método estrutarq.dado.DadoBinario}}

\begin{fulllineitems}
\phantomsection\label{\detokenize{estrutarq.dado:estrutarq.dado.DadoBinario.leia_de_bytes}}
\pysigstartsignatures
\pysiglinewithargsret{\sphinxbfcode{\sphinxupquote{leia\_de\_bytes}}}{\emph{\DUrole{n}{sequencia: bytes) \sphinxhyphen{}\textgreater{} (\textless{}class \textquotesingle{}bytes\textquotesingle{}\textgreater{}}}, \emph{\DUrole{n}{\textless{}class \textquotesingle{}bytes\textquotesingle{}\textgreater{}}}}{}
\pysigstopsignatures
\sphinxAtStartPar
Recuperação de um dado binário de comprimento definido a partir de
uma sequência de bytes.
\begin{quote}\begin{description}
\item[{Parâmetros}] \leavevmode
\sphinxAtStartPar
\sphinxstyleliteralstrong{\sphinxupquote{sequencia}} \textendash{} sequência de bytes

\item[{Retorna}] \leavevmode
\sphinxAtStartPar
tupla com os bytes do dado, removidos os bytes de organização
de dados, e a sequência de bytes restante

\end{description}\end{quote}

\end{fulllineitems}

\index{remova\_formatacao() (método estrutarq.dado.DadoBinario)@\spxentry{remova\_formatacao()}\spxextra{método estrutarq.dado.DadoBinario}}

\begin{fulllineitems}
\phantomsection\label{\detokenize{estrutarq.dado:estrutarq.dado.DadoBinario.remova_formatacao}}
\pysigstartsignatures
\pysiglinewithargsret{\sphinxbfcode{\sphinxupquote{remova\_formatacao}}}{\emph{\DUrole{n}{sequencia}\DUrole{p}{:}\DUrole{w}{  }\DUrole{n}{\sphinxhref{https://docs.python.org/3/library/stdtypes.html\#bytes}{bytes}}}}{{ $\rightarrow$ \sphinxhref{https://docs.python.org/3/library/stdtypes.html\#bytes}{bytes}}}
\pysigstopsignatures
\sphinxAtStartPar
Desformatação do dado: apenas repassa o dado binário.
\begin{quote}\begin{description}
\item[{Parâmetros}] \leavevmode
\sphinxAtStartPar
\sphinxstyleliteralstrong{\sphinxupquote{sequencia}} \textendash{} bytes de dados

\item[{Retorna}] \leavevmode
\sphinxAtStartPar
o dado sem a formatação

\end{description}\end{quote}

\end{fulllineitems}


\end{fulllineitems}



\paragraph{Dado de comprimento fixo}
\label{\detokenize{estrutarq.dado:dado-de-comprimento-fixo}}\index{DadoFixo (classe em estrutarq.dado)@\spxentry{DadoFixo}\spxextra{classe em estrutarq.dado}}

\begin{fulllineitems}
\phantomsection\label{\detokenize{estrutarq.dado:estrutarq.dado.DadoFixo}}
\pysigstartsignatures
\pysiglinewithargsret{\sphinxbfcode{\sphinxupquote{class\DUrole{w}{  }}}\sphinxcode{\sphinxupquote{estrutarq.dado.}}\sphinxbfcode{\sphinxupquote{DadoFixo}}}{\emph{\DUrole{n}{comprimento}\DUrole{p}{:}\DUrole{w}{  }\DUrole{n}{\sphinxhref{https://docs.python.org/3/library/functions.html\#int}{int}}}, \emph{\DUrole{n}{preenchimento}\DUrole{o}{=}\DUrole{default_value}{b\textquotesingle{}\textbackslash{}xff\textquotesingle{}}}}{}
\pysigstopsignatures
\sphinxAtStartPar
Base: {\hyperref[\detokenize{estrutarq.dado:estrutarq.dado.DadoBasico}]{\sphinxcrossref{\sphinxcode{\sphinxupquote{estrutarq.dado.dado\_comum.DadoBasico}}}}}

\sphinxAtStartPar
Classe dado de comprimento fixo
\index{adicione\_formatacao() (método estrutarq.dado.DadoFixo)@\spxentry{adicione\_formatacao()}\spxextra{método estrutarq.dado.DadoFixo}}

\begin{fulllineitems}
\phantomsection\label{\detokenize{estrutarq.dado:estrutarq.dado.DadoFixo.adicione_formatacao}}
\pysigstartsignatures
\pysiglinewithargsret{\sphinxbfcode{\sphinxupquote{adicione\_formatacao}}}{\emph{\DUrole{n}{dado}\DUrole{p}{:}\DUrole{w}{  }\DUrole{n}{\sphinxhref{https://docs.python.org/3/library/stdtypes.html\#bytes}{bytes}}}}{{ $\rightarrow$ \sphinxhref{https://docs.python.org/3/library/stdtypes.html\#bytes}{bytes}}}
\pysigstopsignatures
\sphinxAtStartPar
Formatação do dado: ajusta o dado para o comprimento definido,
truncando ou adicionando o byte de preenchimento.
\begin{quote}\begin{description}
\item[{Parâmetros}] \leavevmode
\sphinxAtStartPar
\sphinxstyleliteralstrong{\sphinxupquote{dado}} \textendash{} valor do dado

\item[{Retorna}] \leavevmode
\sphinxAtStartPar
o dado formatado no comprimento especificado

\end{description}\end{quote}

\end{fulllineitems}

\index{leia\_de\_arquivo() (método estrutarq.dado.DadoFixo)@\spxentry{leia\_de\_arquivo()}\spxextra{método estrutarq.dado.DadoFixo}}

\begin{fulllineitems}
\phantomsection\label{\detokenize{estrutarq.dado:estrutarq.dado.DadoFixo.leia_de_arquivo}}
\pysigstartsignatures
\pysiglinewithargsret{\sphinxbfcode{\sphinxupquote{leia\_de\_arquivo}}}{\emph{\DUrole{n}{arquivo}\DUrole{p}{:}\DUrole{w}{  }\DUrole{n}{\sphinxhref{https://docs.python.org/3/library/typing.html\#typing.BinaryIO}{BinaryIO}}}}{{ $\rightarrow$ \sphinxhref{https://docs.python.org/3/library/stdtypes.html\#bytes}{bytes}}}
\pysigstopsignatures
\sphinxAtStartPar
Leitura de um único dado de comprimento fixo a partir do arquivo
:param arquivo: arquivo binário aberto com permissão de leitura.
\begin{quote}\begin{description}
\item[{Retorna}] \leavevmode
\sphinxAtStartPar
os bytes do campo

\end{description}\end{quote}

\sphinxAtStartPar
Os bytes de preenchimento que existirem são removidos.

\end{fulllineitems}

\index{leia\_de\_bytes() (método estrutarq.dado.DadoFixo)@\spxentry{leia\_de\_bytes()}\spxextra{método estrutarq.dado.DadoFixo}}

\begin{fulllineitems}
\phantomsection\label{\detokenize{estrutarq.dado:estrutarq.dado.DadoFixo.leia_de_bytes}}
\pysigstartsignatures
\pysiglinewithargsret{\sphinxbfcode{\sphinxupquote{leia\_de\_bytes}}}{\emph{\DUrole{n}{sequencia: bytes) \sphinxhyphen{}\textgreater{} (\textless{}class \textquotesingle{}bytes\textquotesingle{}\textgreater{}}}, \emph{\DUrole{n}{\textless{}class \textquotesingle{}bytes\textquotesingle{}\textgreater{}}}}{}
\pysigstopsignatures
\sphinxAtStartPar
Recuperação de um dado individual de uma sequência de bytes,
retornando o dado sem os bytes de preenchimento e o restante da
sequência.
\begin{quote}\begin{description}
\item[{Parâmetros}] \leavevmode
\sphinxAtStartPar
\sphinxstyleliteralstrong{\sphinxupquote{sequencia}} \textendash{} uma sequência de bytes

\item[{Retorna}] \leavevmode
\sphinxAtStartPar
tupla com os bytes do dado, removidos os bytes de organização
de dados, e a sequência de bytes restante

\end{description}\end{quote}

\end{fulllineitems}

\index{preenchimento (propriedade estrutarq.dado.DadoFixo )@\spxentry{preenchimento}\spxextra{propriedade estrutarq.dado.DadoFixo }}

\begin{fulllineitems}
\phantomsection\label{\detokenize{estrutarq.dado:estrutarq.dado.DadoFixo.preenchimento}}
\pysigstartsignatures
\pysigline{\sphinxbfcode{\sphinxupquote{property\DUrole{w}{  }}}\sphinxbfcode{\sphinxupquote{preenchimento}}\sphinxbfcode{\sphinxupquote{\DUrole{p}{:}\DUrole{w}{  }\sphinxhref{https://docs.python.org/3/library/stdtypes.html\#bytes}{bytes}}}}
\pysigstopsignatures
\end{fulllineitems}

\index{remova\_formatacao() (método estrutarq.dado.DadoFixo)@\spxentry{remova\_formatacao()}\spxextra{método estrutarq.dado.DadoFixo}}

\begin{fulllineitems}
\phantomsection\label{\detokenize{estrutarq.dado:estrutarq.dado.DadoFixo.remova_formatacao}}
\pysigstartsignatures
\pysiglinewithargsret{\sphinxbfcode{\sphinxupquote{remova\_formatacao}}}{\emph{\DUrole{n}{sequencia}\DUrole{p}{:}\DUrole{w}{  }\DUrole{n}{\sphinxhref{https://docs.python.org/3/library/stdtypes.html\#bytes}{bytes}}}}{{ $\rightarrow$ \sphinxhref{https://docs.python.org/3/library/stdtypes.html\#bytes}{bytes}}}
\pysigstopsignatures
\sphinxAtStartPar
Desformatação do dado: remoção de caracteres de preenchimento.
\begin{quote}\begin{description}
\item[{Parâmetros}] \leavevmode
\sphinxAtStartPar
\sphinxstyleliteralstrong{\sphinxupquote{sequencia}} \textendash{} bytes de dados

\item[{Retorna}] \leavevmode
\sphinxAtStartPar
dado efetivo, sem preenchimento

\end{description}\end{quote}

\end{fulllineitems}


\end{fulllineitems}



\paragraph{Dado básico}
\label{\detokenize{estrutarq.dado:dado-basico}}
\sphinxAtStartPar
Todas as demais classes do módulo são derivadas de uma classe abstrata básica.
\index{DadoBasico (classe em estrutarq.dado)@\spxentry{DadoBasico}\spxextra{classe em estrutarq.dado}}

\begin{fulllineitems}
\phantomsection\label{\detokenize{estrutarq.dado:estrutarq.dado.DadoBasico}}
\pysigstartsignatures
\pysigline{\sphinxbfcode{\sphinxupquote{class\DUrole{w}{  }}}\sphinxcode{\sphinxupquote{estrutarq.dado.}}\sphinxbfcode{\sphinxupquote{DadoBasico}}}
\pysigstopsignatures
\sphinxAtStartPar
Base: \sphinxhref{https://docs.python.org/3/library/functions.html\#object}{\sphinxcode{\sphinxupquote{object}}}

\sphinxAtStartPar
Classe básica para armazenamento e manipulação de dados.

\sphinxAtStartPar
Implementa as operações básicas e define os métodos abstratos.
\index{adicione\_formatacao() (método estrutarq.dado.DadoBasico)@\spxentry{adicione\_formatacao()}\spxextra{método estrutarq.dado.DadoBasico}}

\begin{fulllineitems}
\phantomsection\label{\detokenize{estrutarq.dado:estrutarq.dado.DadoBasico.adicione_formatacao}}
\pysigstartsignatures
\pysiglinewithargsret{\sphinxbfcode{\sphinxupquote{abstract\DUrole{w}{  }}}\sphinxbfcode{\sphinxupquote{adicione\_formatacao}}}{\emph{\DUrole{n}{dado}\DUrole{p}{:}\DUrole{w}{  }\DUrole{n}{\sphinxhref{https://docs.python.org/3/library/stdtypes.html\#bytes}{bytes}}}}{{ $\rightarrow$ \sphinxhref{https://docs.python.org/3/library/stdtypes.html\#bytes}{bytes}}}
\pysigstopsignatures
\sphinxAtStartPar
Acréscimo da organização de dados em uso aos bytes do dado.
\begin{quote}\begin{description}
\item[{Parâmetros}] \leavevmode
\sphinxAtStartPar
\sphinxstyleliteralstrong{\sphinxupquote{dado}} (\sphinxhref{https://docs.python.org/3/library/stdtypes.html\#bytes}{\sphinxstyleliteralemphasis{\sphinxupquote{bytes}}}) \textendash{} bytes do dado

\item[{Retorna}] \leavevmode
\sphinxAtStartPar
bytes do dado acrescido da forma de organização

\item[{Tipo de retorno}] \leavevmode
\sphinxAtStartPar
\sphinxhref{https://docs.python.org/3/library/stdtypes.html\#bytes}{bytes}

\end{description}\end{quote}

\end{fulllineitems}

\index{byte\_enchimento (atributo estrutarq.dado.DadoBasico)@\spxentry{byte\_enchimento}\spxextra{atributo estrutarq.dado.DadoBasico}}

\begin{fulllineitems}
\phantomsection\label{\detokenize{estrutarq.dado:estrutarq.dado.DadoBasico.byte_enchimento}}
\pysigstartsignatures
\pysigline{\sphinxbfcode{\sphinxupquote{byte\_enchimento}}\sphinxbfcode{\sphinxupquote{\DUrole{p}{:}\DUrole{w}{  }\sphinxhref{https://docs.python.org/3/library/stdtypes.html\#bytes}{bytes}}}\sphinxbfcode{\sphinxupquote{\DUrole{w}{  }\DUrole{p}{=}\DUrole{w}{  }b\textquotesingle{}\textbackslash{}x1b\textquotesingle{}}}}
\pysigstopsignatures
\sphinxAtStartPar
Contém o byte de escape usado para enchimento (\sphinxtitleref{byte stuffing}). Valor
padrão: \sphinxcode{\sphinxupquote{ESC}} (hexadecimal \sphinxcode{\sphinxupquote{0x1B}}).

\end{fulllineitems}

\index{enchimento\_de\_bytes() (método estrutarq.dado.DadoBasico)@\spxentry{enchimento\_de\_bytes()}\spxextra{método estrutarq.dado.DadoBasico}}

\begin{fulllineitems}
\phantomsection\label{\detokenize{estrutarq.dado:estrutarq.dado.DadoBasico.enchimento_de_bytes}}
\pysigstartsignatures
\pysiglinewithargsret{\sphinxbfcode{\sphinxupquote{enchimento\_de\_bytes}}}{\emph{\DUrole{n}{sequencia}\DUrole{p}{:}\DUrole{w}{  }\DUrole{n}{\sphinxhref{https://docs.python.org/3/library/stdtypes.html\#bytes}{bytes}}}, \emph{\DUrole{n}{lista\_bytes}\DUrole{p}{:}\DUrole{w}{  }\DUrole{n}{\sphinxhref{https://docs.python.org/3/library/stdtypes.html\#list}{list}\DUrole{p}{{[}}\sphinxhref{https://docs.python.org/3/library/stdtypes.html\#bytes}{bytes}\DUrole{p}{{]}}}}}{{ $\rightarrow$ \sphinxhref{https://docs.python.org/3/library/stdtypes.html\#bytes}{bytes}}}
\pysigstopsignatures
\sphinxAtStartPar
Operação de enchimento de bytes (\sphinxtitleref{byte stuffing}). Antes de cada item
de \sphinxcode{\sphinxupquote{lista\_bytes}} é acrescentado o byte \sphinxcode{\sphinxupquote{byte\_enchimento}}.
\begin{quote}\begin{description}
\item[{Parâmetros}] \leavevmode\begin{itemize}
\item {} 
\sphinxAtStartPar
\sphinxstyleliteralstrong{\sphinxupquote{sequencia}} (\sphinxhref{https://docs.python.org/3/library/stdtypes.html\#bytes}{\sphinxstyleliteralemphasis{\sphinxupquote{bytes}}}) \textendash{} a sequência de bytes a ser “enchida”

\item {} 
\sphinxAtStartPar
\sphinxstyleliteralstrong{\sphinxupquote{lista\_bytes}} (\sphinxhref{https://docs.python.org/3/library/stdtypes.html\#list}{\sphinxstyleliteralemphasis{\sphinxupquote{list}}}\sphinxstyleliteralemphasis{\sphinxupquote{{[}}}\sphinxhref{https://docs.python.org/3/library/stdtypes.html\#bytes}{\sphinxstyleliteralemphasis{\sphinxupquote{bytes}}}\sphinxstyleliteralemphasis{\sphinxupquote{{]}}}) \textendash{} os bytes especiais que serão “escapados”

\end{itemize}

\item[{Retorna}] \leavevmode
\sphinxAtStartPar
a sequência original enchida

\item[{Tipo de retorno}] \leavevmode
\sphinxAtStartPar
\sphinxhref{https://docs.python.org/3/library/stdtypes.html\#bytes}{bytes}

\end{description}\end{quote}

\end{fulllineitems}

\index{esvaziamento\_de\_bytes() (método estrutarq.dado.DadoBasico)@\spxentry{esvaziamento\_de\_bytes()}\spxextra{método estrutarq.dado.DadoBasico}}

\begin{fulllineitems}
\phantomsection\label{\detokenize{estrutarq.dado:estrutarq.dado.DadoBasico.esvaziamento_de_bytes}}
\pysigstartsignatures
\pysiglinewithargsret{\sphinxbfcode{\sphinxupquote{esvaziamento\_de\_bytes}}}{\emph{\DUrole{n}{sequencia}\DUrole{p}{:}\DUrole{w}{  }\DUrole{n}{\sphinxhref{https://docs.python.org/3/library/stdtypes.html\#bytes}{bytes}}}}{{ $\rightarrow$ \sphinxhref{https://docs.python.org/3/library/stdtypes.html\#bytes}{bytes}}}
\pysigstopsignatures
\sphinxAtStartPar
Operação de esvaziamento de bytes (\sphinxtitleref{byte un\sphinxhyphen{}stuffing}). Todos os
enchimentos feitos com \sphinxcode{\sphinxupquote{byte\_enchimento}} são removidos.
\begin{quote}\begin{description}
\item[{Parâmetros}] \leavevmode
\sphinxAtStartPar
\sphinxstyleliteralstrong{\sphinxupquote{sequencia}} (\sphinxhref{https://docs.python.org/3/library/stdtypes.html\#bytes}{\sphinxstyleliteralemphasis{\sphinxupquote{bytes}}}) \textendash{} a sequência de bytes a ser “esvaziada”

\item[{Retorna}] \leavevmode
\sphinxAtStartPar
a sequência sem os enchimentos

\item[{Tipo de retorno}] \leavevmode
\sphinxAtStartPar
\sphinxhref{https://docs.python.org/3/library/stdtypes.html\#bytes}{bytes}

\end{description}\end{quote}

\end{fulllineitems}

\index{leia\_de\_arquivo() (método estrutarq.dado.DadoBasico)@\spxentry{leia\_de\_arquivo()}\spxextra{método estrutarq.dado.DadoBasico}}

\begin{fulllineitems}
\phantomsection\label{\detokenize{estrutarq.dado:estrutarq.dado.DadoBasico.leia_de_arquivo}}
\pysigstartsignatures
\pysiglinewithargsret{\sphinxbfcode{\sphinxupquote{abstract\DUrole{w}{  }}}\sphinxbfcode{\sphinxupquote{leia\_de\_arquivo}}}{\emph{\DUrole{n}{arquivo}\DUrole{p}{:}\DUrole{w}{  }\DUrole{n}{\sphinxhref{https://docs.python.org/3/library/typing.html\#typing.BinaryIO}{BinaryIO}}}}{{ $\rightarrow$ \sphinxhref{https://docs.python.org/3/library/stdtypes.html\#bytes}{bytes}}}
\pysigstopsignatures
\sphinxAtStartPar
Recuperação de um dado lido de um arquivo, observando a representação
do dado e a forma de organização. A forma de organização usada é
removida.
\begin{quote}\begin{description}
\item[{Parâmetros}] \leavevmode
\sphinxAtStartPar
\sphinxstyleliteralstrong{\sphinxupquote{arquivo}} (\sphinxstyleliteralemphasis{\sphinxupquote{BinaryIO}}) \textendash{} arquivo binário aberto com permissão de leitura

\item[{Retorna}] \leavevmode
\sphinxAtStartPar
a sequência de bytes lida

\item[{Tipo de retorno}] \leavevmode
\sphinxAtStartPar
\sphinxhref{https://docs.python.org/3/library/stdtypes.html\#bytes}{bytes}

\end{description}\end{quote}

\end{fulllineitems}

\index{leia\_de\_bytes() (método estrutarq.dado.DadoBasico)@\spxentry{leia\_de\_bytes()}\spxextra{método estrutarq.dado.DadoBasico}}

\begin{fulllineitems}
\phantomsection\label{\detokenize{estrutarq.dado:estrutarq.dado.DadoBasico.leia_de_bytes}}
\pysigstartsignatures
\pysiglinewithargsret{\sphinxbfcode{\sphinxupquote{abstract\DUrole{w}{  }}}\sphinxbfcode{\sphinxupquote{leia\_de\_bytes}}}{\emph{\DUrole{n}{sequencia}\DUrole{p}{:}\DUrole{w}{  }\DUrole{n}{\sphinxhref{https://docs.python.org/3/library/stdtypes.html\#bytes}{bytes}}}}{{ $\rightarrow$ \sphinxhref{https://docs.python.org/3/library/stdtypes.html\#tuple}{tuple}\DUrole{p}{{[}}\sphinxhref{https://docs.python.org/3/library/stdtypes.html\#bytes}{bytes}\DUrole{p}{,}\DUrole{w}{  }\sphinxhref{https://docs.python.org/3/library/stdtypes.html\#bytes}{bytes}\DUrole{p}{{]}}}}
\pysigstopsignatures
\sphinxAtStartPar
Recuperação de um dado a partir de uma sequência de bytes, retornando
os bytes do dado em si e o restante da sequência depois da extração
do dado, observando a representação do dado e a forma de organização.
O dado é retornado sem a organização.
\begin{quote}\begin{description}
\item[{Parâmetros}] \leavevmode
\sphinxAtStartPar
\sphinxstyleliteralstrong{\sphinxupquote{sequencia}} (\sphinxhref{https://docs.python.org/3/library/stdtypes.html\#bytes}{\sphinxstyleliteralemphasis{\sphinxupquote{bytes}}}) \textendash{} sequência de bytes

\item[{Retorna}] \leavevmode
\sphinxAtStartPar
tupla com os bytes do dado, removidos os bytes de organização
de dados, e a sequência de bytes restante

\item[{Tipo de retorno}] \leavevmode
\sphinxAtStartPar
\sphinxhref{https://docs.python.org/3/library/stdtypes.html\#tuple}{tuple}{[}\sphinxhref{https://docs.python.org/3/library/stdtypes.html\#bytes}{bytes}, \sphinxhref{https://docs.python.org/3/library/stdtypes.html\#bytes}{bytes}{]}

\end{description}\end{quote}

\end{fulllineitems}

\index{remova\_formatacao() (método estrutarq.dado.DadoBasico)@\spxentry{remova\_formatacao()}\spxextra{método estrutarq.dado.DadoBasico}}

\begin{fulllineitems}
\phantomsection\label{\detokenize{estrutarq.dado:estrutarq.dado.DadoBasico.remova_formatacao}}
\pysigstartsignatures
\pysiglinewithargsret{\sphinxbfcode{\sphinxupquote{abstract\DUrole{w}{  }}}\sphinxbfcode{\sphinxupquote{remova\_formatacao}}}{\emph{\DUrole{n}{sequencia}\DUrole{p}{:}\DUrole{w}{  }\DUrole{n}{\sphinxhref{https://docs.python.org/3/library/stdtypes.html\#bytes}{bytes}}}}{{ $\rightarrow$ \sphinxhref{https://docs.python.org/3/library/stdtypes.html\#bytes}{bytes}}}
\pysigstopsignatures
\sphinxAtStartPar
Remoção dos bytes correspondentes à forma de organização da sequência
de bytes.
\begin{quote}\begin{description}
\item[{Parâmetros}] \leavevmode
\sphinxAtStartPar
\sphinxstyleliteralstrong{\sphinxupquote{sequencia}} (\sphinxhref{https://docs.python.org/3/library/stdtypes.html\#bytes}{\sphinxstyleliteralemphasis{\sphinxupquote{bytes}}}) \textendash{} uma sequência de bytes

\item[{Retorna}] \leavevmode
\sphinxAtStartPar
a sequência após extraídos os bytes de organização

\item[{Tipo de retorno}] \leavevmode
\sphinxAtStartPar
\sphinxhref{https://docs.python.org/3/library/stdtypes.html\#bytes}{bytes}

\end{description}\end{quote}

\end{fulllineitems}

\index{varredura\_com\_enchimento() (método estrutarq.dado.DadoBasico)@\spxentry{varredura\_com\_enchimento()}\spxextra{método estrutarq.dado.DadoBasico}}

\begin{fulllineitems}
\phantomsection\label{\detokenize{estrutarq.dado:estrutarq.dado.DadoBasico.varredura_com_enchimento}}
\pysigstartsignatures
\pysiglinewithargsret{\sphinxbfcode{\sphinxupquote{varredura\_com\_enchimento}}}{\emph{\DUrole{n}{sequencia}\DUrole{p}{:}\DUrole{w}{  }\DUrole{n}{\sphinxhref{https://docs.python.org/3/library/stdtypes.html\#bytes}{bytes}}}, \emph{\DUrole{n}{referencia}\DUrole{p}{:}\DUrole{w}{  }\DUrole{n}{\sphinxhref{https://docs.python.org/3/library/stdtypes.html\#bytes}{bytes}}}}{{ $\rightarrow$ \sphinxhref{https://docs.python.org/3/library/stdtypes.html\#tuple}{tuple}\DUrole{p}{{[}}\sphinxhref{https://docs.python.org/3/library/stdtypes.html\#bytes}{bytes}\DUrole{p}{,}\DUrole{w}{  }\sphinxhref{https://docs.python.org/3/library/stdtypes.html\#bytes}{bytes}\DUrole{p}{{]}}}}
\pysigstopsignatures
\sphinxAtStartPar
Recuperação de um dado individual de uma sequência de bytes,
retornando o dado até um byte de referência (não “enchido”) e o
restante da sequência depois desse byte.
\begin{quote}\begin{description}
\item[{Parâmetros}] \leavevmode\begin{itemize}
\item {} 
\sphinxAtStartPar
\sphinxstyleliteralstrong{\sphinxupquote{sequencia}} (\sphinxhref{https://docs.python.org/3/library/stdtypes.html\#bytes}{\sphinxstyleliteralemphasis{\sphinxupquote{bytes}}}) \textendash{} uma sequência de bytes

\item {} 
\sphinxAtStartPar
\sphinxstyleliteralstrong{\sphinxupquote{referencia}} (\sphinxhref{https://docs.python.org/3/library/stdtypes.html\#bytes}{\sphinxstyleliteralemphasis{\sphinxupquote{bytes}}}) \textendash{} byte simples usado como sentinela (terminador)

\end{itemize}

\item[{Retorna}] \leavevmode
\sphinxAtStartPar
uma tupla contendo a sequência de bytes até \sphinxcode{\sphinxupquote{referencia}}
e o restante da sequência depois de \sphinxcode{\sphinxupquote{referencia}}

\item[{Tipo de retorno}] \leavevmode
\sphinxAtStartPar
\sphinxhref{https://docs.python.org/3/library/stdtypes.html\#tuple}{tuple}{[}\sphinxhref{https://docs.python.org/3/library/stdtypes.html\#bytes}{bytes}, \sphinxhref{https://docs.python.org/3/library/stdtypes.html\#bytes}{bytes}{]}

\item[{Levanta}] \leavevmode
\sphinxAtStartPar
\sphinxhref{https://docs.python.org/3/library/exceptions.html\#ValueError}{\sphinxstyleliteralstrong{\sphinxupquote{ValueError}}} \textendash{} se o byte de referência não estiver presente na
sequência de bytes

\end{description}\end{quote}

\end{fulllineitems}


\end{fulllineitems}


\sphinxstepscope


\subsection{estrutarq.campo package}
\label{\detokenize{estrutarq.campo:estrutarq-campo-package}}\label{\detokenize{estrutarq.campo::doc}}

\subsubsection{Submodules}
\label{\detokenize{estrutarq.campo:submodules}}

\subsubsection{estrutarq.campo.campo\_cadeia module}
\label{\detokenize{estrutarq.campo:module-estrutarq.campo.campo_cadeia}}\label{\detokenize{estrutarq.campo:estrutarq-campo-campo-cadeia-module}}\index{módulo@\spxentry{módulo}!estrutarq.campo.campo\_cadeia@\spxentry{estrutarq.campo.campo\_cadeia}}\index{estrutarq.campo.campo\_cadeia@\spxentry{estrutarq.campo.campo\_cadeia}!módulo@\spxentry{módulo}}
\sphinxAtStartPar
Campos para armazenamento de cadeias de caracteres.

\sphinxAtStartPar
Este arquivo provê classes para uso de campos cujo conteúdo é
uma cadeia de caracteres. Internamente, o tipo \sphinxtitleref{str} é usado
para armazenamento e a transformação para sequência de bytes
usa a codificação UTF\sphinxhyphen{}8.

\sphinxAtStartPar
Uma classe básica {\hyperref[\detokenize{estrutarq.campo:estrutarq.campo.campo_cadeia.CampoCadeiaBasico}]{\sphinxcrossref{\sphinxcode{\sphinxupquote{CampoCadeiaBasico}}}}} define uma classe
abstrata (ABC) com as propriedades e métodos gerais. Dela são
derivadas campos:
\begin{itemize}
\item {} 
\sphinxAtStartPar
Com terminadores

\item {} 
\sphinxAtStartPar
Prefixada pelo comprimento

\item {} 
\sphinxAtStartPar
De comprimento fixo predefinido

\end{itemize}

\sphinxAtStartPar
Licença: GNU GENERAL PUBLIC LICENSE V.3, 2007

\sphinxAtStartPar
Jander Moreira, 2021\sphinxhyphen{}2022
\index{CampoCadeiaBasico (classe em estrutarq.campo.campo\_cadeia)@\spxentry{CampoCadeiaBasico}\spxextra{classe em estrutarq.campo.campo\_cadeia}}

\begin{fulllineitems}
\phantomsection\label{\detokenize{estrutarq.campo:estrutarq.campo.campo_cadeia.CampoCadeiaBasico}}
\pysigstartsignatures
\pysiglinewithargsret{\sphinxbfcode{\sphinxupquote{class\DUrole{w}{  }}}\sphinxcode{\sphinxupquote{estrutarq.campo.campo\_cadeia.}}\sphinxbfcode{\sphinxupquote{CampoCadeiaBasico}}}{\emph{\DUrole{n}{tipo}\DUrole{p}{:}\DUrole{w}{  }\DUrole{n}{\sphinxhref{https://docs.python.org/3/library/stdtypes.html\#str}{str}}}, \emph{\DUrole{n}{valor}\DUrole{p}{:}\DUrole{w}{  }\DUrole{n}{\sphinxhref{https://docs.python.org/3/library/stdtypes.html\#str}{str}}\DUrole{w}{  }\DUrole{o}{=}\DUrole{w}{  }\DUrole{default_value}{\textquotesingle{}\textquotesingle{}}}}{}
\pysigstopsignatures
\sphinxAtStartPar
Base: {\hyperref[\detokenize{estrutarq.campo:estrutarq.campo.campo_comum.CampoBasico}]{\sphinxcrossref{\sphinxcode{\sphinxupquote{estrutarq.campo.campo\_comum.CampoBasico}}}}}

\sphinxAtStartPar
Classe básica para cadeias de caracteres.
\begin{quote}\begin{description}
\item[{Parâmetros}] \leavevmode\begin{itemize}
\item {} 
\sphinxAtStartPar
\sphinxstyleliteralstrong{\sphinxupquote{tipo}} (\sphinxhref{https://docs.python.org/3/library/stdtypes.html\#str}{\sphinxstyleliteralemphasis{\sphinxupquote{str}}}) \textendash{} nome do tipo (definido nas classes derivadas)

\item {} 
\sphinxAtStartPar
\sphinxstyleliteralstrong{\sphinxupquote{valor}} (\sphinxhref{https://docs.python.org/3/library/stdtypes.html\#str}{\sphinxstyleliteralemphasis{\sphinxupquote{str}}}\sphinxstyleliteralemphasis{\sphinxupquote{, }}\sphinxstyleliteralemphasis{\sphinxupquote{opcional}}) \textendash{} o valor a ser armazenado no campo (padrão: \sphinxcode{\sphinxupquote{""}})

\end{itemize}

\end{description}\end{quote}
\index{bytes\_para\_valor() (método estrutarq.campo.campo\_cadeia.CampoCadeiaBasico)@\spxentry{bytes\_para\_valor()}\spxextra{método estrutarq.campo.campo\_cadeia.CampoCadeiaBasico}}

\begin{fulllineitems}
\phantomsection\label{\detokenize{estrutarq.campo:estrutarq.campo.campo_cadeia.CampoCadeiaBasico.bytes_para_valor}}
\pysigstartsignatures
\pysiglinewithargsret{\sphinxbfcode{\sphinxupquote{bytes\_para\_valor}}}{\emph{\DUrole{n}{dado}\DUrole{p}{:}\DUrole{w}{  }\DUrole{n}{\sphinxhref{https://docs.python.org/3/library/stdtypes.html\#bytes}{bytes}}}}{}
\pysigstopsignatures
\sphinxAtStartPar
Armazenamento da sequência de bytes de \sphinxcode{\sphinxupquote{dado}} como valor
do campo.
\begin{quote}\begin{description}
\item[{Parâmetros}] \leavevmode
\sphinxAtStartPar
\sphinxstyleliteralstrong{\sphinxupquote{dado}} (\sphinxhref{https://docs.python.org/3/library/stdtypes.html\#bytes}{\sphinxstyleliteralemphasis{\sphinxupquote{bytes}}}) \textendash{} sequência de bytes com codificação UTF\sphinxhyphen{}8

\end{description}\end{quote}

\end{fulllineitems}

\index{valor (propriedade estrutarq.campo.campo\_cadeia.CampoCadeiaBasico )@\spxentry{valor}\spxextra{propriedade estrutarq.campo.campo\_cadeia.CampoCadeiaBasico }}

\begin{fulllineitems}
\phantomsection\label{\detokenize{estrutarq.campo:estrutarq.campo.campo_cadeia.CampoCadeiaBasico.valor}}
\pysigstartsignatures
\pysigline{\sphinxbfcode{\sphinxupquote{property\DUrole{w}{  }}}\sphinxbfcode{\sphinxupquote{valor}}}
\pysigstopsignatures
\sphinxAtStartPar
Recuperação, com as devidas conversões, do atributo \sphinxcode{\sphinxupquote{\_\_valor}}
:return: o valor de \sphinxcode{\sphinxupquote{\_\_valor}}

\end{fulllineitems}

\index{valor\_para\_bytes() (método estrutarq.campo.campo\_cadeia.CampoCadeiaBasico)@\spxentry{valor\_para\_bytes()}\spxextra{método estrutarq.campo.campo\_cadeia.CampoCadeiaBasico}}

\begin{fulllineitems}
\phantomsection\label{\detokenize{estrutarq.campo:estrutarq.campo.campo_cadeia.CampoCadeiaBasico.valor_para_bytes}}
\pysigstartsignatures
\pysiglinewithargsret{\sphinxbfcode{\sphinxupquote{valor\_para\_bytes}}}{}{{ $\rightarrow$ \sphinxhref{https://docs.python.org/3/library/stdtypes.html\#bytes}{bytes}}}
\pysigstopsignatures
\sphinxAtStartPar
Retorno do valor do campo convertido para sequência
de bytes usando codificação UTF\sphinxhyphen{}8.
\begin{quote}\begin{description}
\item[{Retorna}] \leavevmode
\sphinxAtStartPar
sequência de bytes

\item[{Tipo de retorno}] \leavevmode
\sphinxAtStartPar
\sphinxhref{https://docs.python.org/3/library/stdtypes.html\#bytes}{bytes}

\end{description}\end{quote}

\end{fulllineitems}


\end{fulllineitems}

\index{CampoCadeiaFixo (classe em estrutarq.campo.campo\_cadeia)@\spxentry{CampoCadeiaFixo}\spxextra{classe em estrutarq.campo.campo\_cadeia}}

\begin{fulllineitems}
\phantomsection\label{\detokenize{estrutarq.campo:estrutarq.campo.campo_cadeia.CampoCadeiaFixo}}
\pysigstartsignatures
\pysiglinewithargsret{\sphinxbfcode{\sphinxupquote{class\DUrole{w}{  }}}\sphinxcode{\sphinxupquote{estrutarq.campo.campo\_cadeia.}}\sphinxbfcode{\sphinxupquote{CampoCadeiaFixo}}}{\emph{\DUrole{n}{comprimento}\DUrole{p}{:}\DUrole{w}{  }\DUrole{n}{\sphinxhref{https://docs.python.org/3/library/functions.html\#int}{int}}}, \emph{\DUrole{o}{**}\DUrole{n}{kwargs}}}{}
\pysigstopsignatures
\sphinxAtStartPar
Base: {\hyperref[\detokenize{estrutarq.dado:estrutarq.dado.DadoFixo}]{\sphinxcrossref{\sphinxcode{\sphinxupquote{estrutarq.dado.dado\_comum.DadoFixo}}}}}, {\hyperref[\detokenize{estrutarq.campo:estrutarq.campo.campo_cadeia.CampoCadeiaBasico}]{\sphinxcrossref{\sphinxcode{\sphinxupquote{estrutarq.campo.campo\_cadeia.CampoCadeiaBasico}}}}}

\sphinxAtStartPar
Classe para cadeia de caracteres com comprimento\_bloco fixo e preenchimento
de dados inválidos

\end{fulllineitems}

\index{CampoCadeiaPrefixado (classe em estrutarq.campo.campo\_cadeia)@\spxentry{CampoCadeiaPrefixado}\spxextra{classe em estrutarq.campo.campo\_cadeia}}

\begin{fulllineitems}
\phantomsection\label{\detokenize{estrutarq.campo:estrutarq.campo.campo_cadeia.CampoCadeiaPrefixado}}
\pysigstartsignatures
\pysiglinewithargsret{\sphinxbfcode{\sphinxupquote{class\DUrole{w}{  }}}\sphinxcode{\sphinxupquote{estrutarq.campo.campo\_cadeia.}}\sphinxbfcode{\sphinxupquote{CampoCadeiaPrefixado}}}{\emph{\DUrole{o}{*}\DUrole{n}{args}}, \emph{\DUrole{o}{**}\DUrole{n}{kwargs}}}{}
\pysigstopsignatures
\sphinxAtStartPar
Base: {\hyperref[\detokenize{estrutarq.dado:estrutarq.dado.DadoPrefixado}]{\sphinxcrossref{\sphinxcode{\sphinxupquote{estrutarq.dado.dado\_comum.DadoPrefixado}}}}}, {\hyperref[\detokenize{estrutarq.campo:estrutarq.campo.campo_cadeia.CampoCadeiaBasico}]{\sphinxcrossref{\sphinxcode{\sphinxupquote{estrutarq.campo.campo\_cadeia.CampoCadeiaBasico}}}}}

\sphinxAtStartPar
Classe para cadeia de caracteres prefixada pelo comprimento\_bloco

\end{fulllineitems}

\index{CampoCadeiaTerminador (classe em estrutarq.campo.campo\_cadeia)@\spxentry{CampoCadeiaTerminador}\spxextra{classe em estrutarq.campo.campo\_cadeia}}

\begin{fulllineitems}
\phantomsection\label{\detokenize{estrutarq.campo:estrutarq.campo.campo_cadeia.CampoCadeiaTerminador}}
\pysigstartsignatures
\pysiglinewithargsret{\sphinxbfcode{\sphinxupquote{class\DUrole{w}{  }}}\sphinxcode{\sphinxupquote{estrutarq.campo.campo\_cadeia.}}\sphinxbfcode{\sphinxupquote{CampoCadeiaTerminador}}}{\emph{\DUrole{o}{**}\DUrole{n}{kwargs}}}{}
\pysigstopsignatures
\sphinxAtStartPar
Base: {\hyperref[\detokenize{estrutarq.dado:estrutarq.dado.DadoTerminador}]{\sphinxcrossref{\sphinxcode{\sphinxupquote{estrutarq.dado.dado\_comum.DadoTerminador}}}}}, {\hyperref[\detokenize{estrutarq.campo:estrutarq.campo.campo_cadeia.CampoCadeiaBasico}]{\sphinxcrossref{\sphinxcode{\sphinxupquote{estrutarq.campo.campo\_cadeia.CampoCadeiaBasico}}}}}

\sphinxAtStartPar
Classe para cadeia de caracteres com terminador
\begin{quote}\begin{description}
\item[{Parâmetros}] \leavevmode
\sphinxAtStartPar
\sphinxstyleliteralstrong{\sphinxupquote{kwargs}} (\sphinxstyleliteralemphasis{\sphinxupquote{:class:dict}}) \textendash{} parâmetros nomeados a serem repassados

\end{description}\end{quote}

\end{fulllineitems}



\subsubsection{estrutarq.campo.campo\_comum module}
\label{\detokenize{estrutarq.campo:module-estrutarq.campo.campo_comum}}\label{\detokenize{estrutarq.campo:estrutarq-campo-campo-comum-module}}\index{módulo@\spxentry{módulo}!estrutarq.campo.campo\_comum@\spxentry{estrutarq.campo.campo\_comum}}\index{estrutarq.campo.campo\_comum@\spxentry{estrutarq.campo.campo\_comum}!módulo@\spxentry{módulo}}\index{CampoBasico (classe em estrutarq.campo.campo\_comum)@\spxentry{CampoBasico}\spxextra{classe em estrutarq.campo.campo\_comum}}

\begin{fulllineitems}
\phantomsection\label{\detokenize{estrutarq.campo:estrutarq.campo.campo_comum.CampoBasico}}
\pysigstartsignatures
\pysiglinewithargsret{\sphinxbfcode{\sphinxupquote{class\DUrole{w}{  }}}\sphinxcode{\sphinxupquote{estrutarq.campo.campo\_comum.}}\sphinxbfcode{\sphinxupquote{CampoBasico}}}{\emph{\DUrole{n}{tipo}\DUrole{p}{:}\DUrole{w}{  }\DUrole{n}{\sphinxhref{https://docs.python.org/3/library/stdtypes.html\#str}{str}}}}{}
\pysigstopsignatures
\sphinxAtStartPar
Base: {\hyperref[\detokenize{estrutarq.dado:estrutarq.dado.DadoBasico}]{\sphinxcrossref{\sphinxcode{\sphinxupquote{estrutarq.dado.dado\_comum.DadoBasico}}}}}

\sphinxAtStartPar
Estruturação básica do campo como menor unidade de informação.
\begin{quote}\begin{description}
\item[{Parâmetros}] \leavevmode
\sphinxAtStartPar
\sphinxstyleliteralstrong{\sphinxupquote{tipo}} (\sphinxhref{https://docs.python.org/3/library/stdtypes.html\#str}{\sphinxstyleliteralemphasis{\sphinxupquote{str}}}) \textendash{} cadeia de caracteres com o nome do tipo

\end{description}\end{quote}
\index{bytes\_para\_valor() (método estrutarq.campo.campo\_comum.CampoBasico)@\spxentry{bytes\_para\_valor()}\spxextra{método estrutarq.campo.campo\_comum.CampoBasico}}

\begin{fulllineitems}
\phantomsection\label{\detokenize{estrutarq.campo:estrutarq.campo.campo_comum.CampoBasico.bytes_para_valor}}
\pysigstartsignatures
\pysiglinewithargsret{\sphinxbfcode{\sphinxupquote{abstract\DUrole{w}{  }}}\sphinxbfcode{\sphinxupquote{bytes\_para\_valor}}}{\emph{\DUrole{n}{dado}\DUrole{p}{:}\DUrole{w}{  }\DUrole{n}{\sphinxhref{https://docs.python.org/3/library/stdtypes.html\#bytes}{bytes}}}}{}
\pysigstopsignatures
\sphinxAtStartPar
Conversão de uma sequência de bytes para armazenamento no valor
do campo, de acordo com a representação de dados
:param dado: sequência de bytes
:return: o valor do campo de acordo com seu tipo

\end{fulllineitems}

\index{comprimento() (método estrutarq.campo.campo\_comum.CampoBasico)@\spxentry{comprimento()}\spxextra{método estrutarq.campo.campo\_comum.CampoBasico}}

\begin{fulllineitems}
\phantomsection\label{\detokenize{estrutarq.campo:estrutarq.campo.campo_comum.CampoBasico.comprimento}}
\pysigstartsignatures
\pysiglinewithargsret{\sphinxbfcode{\sphinxupquote{comprimento}}}{}{}
\pysigstopsignatures
\sphinxAtStartPar
Obtém o comprimento atual do campo
\begin{quote}\begin{description}
\item[{Retorna}] \leavevmode
\sphinxAtStartPar
o comprimento do campo

\end{description}\end{quote}

\end{fulllineitems}

\index{comprimento\_fixo() (método estrutarq.campo.campo\_comum.CampoBasico)@\spxentry{comprimento\_fixo()}\spxextra{método estrutarq.campo.campo\_comum.CampoBasico}}

\begin{fulllineitems}
\phantomsection\label{\detokenize{estrutarq.campo:estrutarq.campo.campo_comum.CampoBasico.comprimento_fixo}}
\pysigstartsignatures
\pysiglinewithargsret{\sphinxbfcode{\sphinxupquote{comprimento\_fixo}}}{}{}
\pysigstopsignatures
\sphinxAtStartPar
Retorna se o comprimento é ou não fixo
\begin{quote}\begin{description}
\item[{Retorna}] \leavevmode
\sphinxAtStartPar
\sphinxtitleref{True} para comprimento fixo ou \sphinxtitleref{False} para variável

\end{description}\end{quote}

\end{fulllineitems}

\index{copy() (método estrutarq.campo.campo\_comum.CampoBasico)@\spxentry{copy()}\spxextra{método estrutarq.campo.campo\_comum.CampoBasico}}

\begin{fulllineitems}
\phantomsection\label{\detokenize{estrutarq.campo:estrutarq.campo.campo_comum.CampoBasico.copy}}
\pysigstartsignatures
\pysiglinewithargsret{\sphinxbfcode{\sphinxupquote{copy}}}{}{}
\pysigstopsignatures
\sphinxAtStartPar
Cópia “rasa” deste campo
:return: outra instância com os mesmos valores

\end{fulllineitems}

\index{escreva() (método estrutarq.campo.campo\_comum.CampoBasico)@\spxentry{escreva()}\spxextra{método estrutarq.campo.campo\_comum.CampoBasico}}

\begin{fulllineitems}
\phantomsection\label{\detokenize{estrutarq.campo:estrutarq.campo.campo_comum.CampoBasico.escreva}}
\pysigstartsignatures
\pysiglinewithargsret{\sphinxbfcode{\sphinxupquote{escreva}}}{\emph{\DUrole{n}{arquivo}\DUrole{p}{:}\DUrole{w}{  }\DUrole{n}{\sphinxhref{https://docs.python.org/3/library/typing.html\#typing.BinaryIO}{BinaryIO}}}}{}
\pysigstopsignatures
\sphinxAtStartPar
Conversão do valor para sequência de bytes e armazenamento no
arquivo
\begin{quote}\begin{description}
\item[{Parâmetros}] \leavevmode
\sphinxAtStartPar
\sphinxstyleliteralstrong{\sphinxupquote{arquivo}} \textendash{} arquivo binário aberto com permissão de escrita

\end{description}\end{quote}

\end{fulllineitems}

\index{leia() (método estrutarq.campo.campo\_comum.CampoBasico)@\spxentry{leia()}\spxextra{método estrutarq.campo.campo\_comum.CampoBasico}}

\begin{fulllineitems}
\phantomsection\label{\detokenize{estrutarq.campo:estrutarq.campo.campo_comum.CampoBasico.leia}}
\pysigstartsignatures
\pysiglinewithargsret{\sphinxbfcode{\sphinxupquote{leia}}}{\emph{\DUrole{n}{arquivo}\DUrole{p}{:}\DUrole{w}{  }\DUrole{n}{\sphinxhref{https://docs.python.org/3/library/typing.html\#typing.BinaryIO}{BinaryIO}}}}{}
\pysigstopsignatures
\sphinxAtStartPar
Conversão dos dado lidos para o valor do campo, obedecendo à
organização e formato de representação
\begin{quote}\begin{description}
\item[{Parâmetros}] \leavevmode
\sphinxAtStartPar
\sphinxstyleliteralstrong{\sphinxupquote{arquivo}} \textendash{} arquivo binário aberto com permissão de leitura

\end{description}\end{quote}

\end{fulllineitems}

\index{tipo (propriedade estrutarq.campo.campo\_comum.CampoBasico )@\spxentry{tipo}\spxextra{propriedade estrutarq.campo.campo\_comum.CampoBasico }}

\begin{fulllineitems}
\phantomsection\label{\detokenize{estrutarq.campo:estrutarq.campo.campo_comum.CampoBasico.tipo}}
\pysigstartsignatures
\pysigline{\sphinxbfcode{\sphinxupquote{property\DUrole{w}{  }}}\sphinxbfcode{\sphinxupquote{tipo}}}
\pysigstopsignatures
\end{fulllineitems}

\index{valor (propriedade estrutarq.campo.campo\_comum.CampoBasico )@\spxentry{valor}\spxextra{propriedade estrutarq.campo.campo\_comum.CampoBasico }}

\begin{fulllineitems}
\phantomsection\label{\detokenize{estrutarq.campo:estrutarq.campo.campo_comum.CampoBasico.valor}}
\pysigstartsignatures
\pysigline{\sphinxbfcode{\sphinxupquote{abstract\DUrole{w}{  }property\DUrole{w}{  }}}\sphinxbfcode{\sphinxupquote{valor}}}
\pysigstopsignatures
\sphinxAtStartPar
Recuperação, com as devidas conversões, do atributo \sphinxcode{\sphinxupquote{\_\_valor}}
:return: o valor de \sphinxcode{\sphinxupquote{\_\_valor}}

\end{fulllineitems}

\index{valor\_para\_bytes() (método estrutarq.campo.campo\_comum.CampoBasico)@\spxentry{valor\_para\_bytes()}\spxextra{método estrutarq.campo.campo\_comum.CampoBasico}}

\begin{fulllineitems}
\phantomsection\label{\detokenize{estrutarq.campo:estrutarq.campo.campo_comum.CampoBasico.valor_para_bytes}}
\pysigstartsignatures
\pysiglinewithargsret{\sphinxbfcode{\sphinxupquote{abstract\DUrole{w}{  }}}\sphinxbfcode{\sphinxupquote{valor\_para\_bytes}}}{}{{ $\rightarrow$ \sphinxhref{https://docs.python.org/3/library/stdtypes.html\#bytes}{bytes}}}
\pysigstopsignatures
\sphinxAtStartPar
Conversão do valor do campo para sequência de bytes de acordo
com a representação de dados
:return:

\end{fulllineitems}


\end{fulllineitems}

\index{CampoBruto (classe em estrutarq.campo.campo\_comum)@\spxentry{CampoBruto}\spxextra{classe em estrutarq.campo.campo\_comum}}

\begin{fulllineitems}
\phantomsection\label{\detokenize{estrutarq.campo:estrutarq.campo.campo_comum.CampoBruto}}
\pysigstartsignatures
\pysiglinewithargsret{\sphinxbfcode{\sphinxupquote{class\DUrole{w}{  }}}\sphinxcode{\sphinxupquote{estrutarq.campo.campo\_comum.}}\sphinxbfcode{\sphinxupquote{CampoBruto}}}{\emph{\DUrole{n}{valor}\DUrole{o}{=}\DUrole{default_value}{\textquotesingle{}\textquotesingle{}}}}{}
\pysigstopsignatures
\sphinxAtStartPar
Base: {\hyperref[\detokenize{estrutarq.dado:estrutarq.dado.DadoBruto}]{\sphinxcrossref{\sphinxcode{\sphinxupquote{estrutarq.dado.dado\_comum.DadoBruto}}}}}, {\hyperref[\detokenize{estrutarq.campo:estrutarq.campo.campo_comum.CampoBasico}]{\sphinxcrossref{\sphinxcode{\sphinxupquote{estrutarq.campo.campo\_comum.CampoBasico}}}}}

\sphinxAtStartPar
Implementação das funções de um campo bruto, ou seja, sem organização
de campo. O valor é sempre armazenado como cadeia de caracteres.
\index{bytes\_para\_valor() (método estrutarq.campo.campo\_comum.CampoBruto)@\spxentry{bytes\_para\_valor()}\spxextra{método estrutarq.campo.campo\_comum.CampoBruto}}

\begin{fulllineitems}
\phantomsection\label{\detokenize{estrutarq.campo:estrutarq.campo.campo_comum.CampoBruto.bytes_para_valor}}
\pysigstartsignatures
\pysiglinewithargsret{\sphinxbfcode{\sphinxupquote{bytes\_para\_valor}}}{\emph{\DUrole{n}{dado}\DUrole{p}{:}\DUrole{w}{  }\DUrole{n}{\sphinxhref{https://docs.python.org/3/library/stdtypes.html\#bytes}{bytes}}}}{}
\pysigstopsignatures
\sphinxAtStartPar
Conversão de sequência de bytes para valor o campo, considerando
uma cadeia de caracteres simples
:param dado: a sequência de bytes

\end{fulllineitems}

\index{valor (propriedade estrutarq.campo.campo\_comum.CampoBruto )@\spxentry{valor}\spxextra{propriedade estrutarq.campo.campo\_comum.CampoBruto }}

\begin{fulllineitems}
\phantomsection\label{\detokenize{estrutarq.campo:estrutarq.campo.campo_comum.CampoBruto.valor}}
\pysigstartsignatures
\pysigline{\sphinxbfcode{\sphinxupquote{property\DUrole{w}{  }}}\sphinxbfcode{\sphinxupquote{valor}}}
\pysigstopsignatures
\sphinxAtStartPar
Recuperação, com as devidas conversões, do atributo \sphinxcode{\sphinxupquote{\_\_valor}}
:return: o valor de \sphinxcode{\sphinxupquote{\_\_valor}}

\end{fulllineitems}

\index{valor\_para\_bytes() (método estrutarq.campo.campo\_comum.CampoBruto)@\spxentry{valor\_para\_bytes()}\spxextra{método estrutarq.campo.campo\_comum.CampoBruto}}

\begin{fulllineitems}
\phantomsection\label{\detokenize{estrutarq.campo:estrutarq.campo.campo_comum.CampoBruto.valor_para_bytes}}
\pysigstartsignatures
\pysiglinewithargsret{\sphinxbfcode{\sphinxupquote{valor\_para\_bytes}}}{}{{ $\rightarrow$ \sphinxhref{https://docs.python.org/3/library/stdtypes.html\#bytes}{bytes}}}
\pysigstopsignatures
\sphinxAtStartPar
Conversão do valor do campo (cadeia de caracteres) para uma
sequência de bytes, usando codificação UTF\sphinxhyphen{}8
:return: a sequência de bytes

\end{fulllineitems}


\end{fulllineitems}



\subsubsection{estrutarq.campo.campo\_inteiro module}
\label{\detokenize{estrutarq.campo:module-estrutarq.campo.campo_inteiro}}\label{\detokenize{estrutarq.campo:estrutarq-campo-campo-inteiro-module}}\index{módulo@\spxentry{módulo}!estrutarq.campo.campo\_inteiro@\spxentry{estrutarq.campo.campo\_inteiro}}\index{estrutarq.campo.campo\_inteiro@\spxentry{estrutarq.campo.campo\_inteiro}!módulo@\spxentry{módulo}}\index{CampoIntBasico (classe em estrutarq.campo.campo\_inteiro)@\spxentry{CampoIntBasico}\spxextra{classe em estrutarq.campo.campo\_inteiro}}

\begin{fulllineitems}
\phantomsection\label{\detokenize{estrutarq.campo:estrutarq.campo.campo_inteiro.CampoIntBasico}}
\pysigstartsignatures
\pysiglinewithargsret{\sphinxbfcode{\sphinxupquote{class\DUrole{w}{  }}}\sphinxcode{\sphinxupquote{estrutarq.campo.campo\_inteiro.}}\sphinxbfcode{\sphinxupquote{CampoIntBasico}}}{\emph{\DUrole{n}{tipo}\DUrole{p}{:}\DUrole{w}{  }\DUrole{n}{\sphinxhref{https://docs.python.org/3/library/stdtypes.html\#str}{str}}}, \emph{\DUrole{n}{valor}\DUrole{p}{:}\DUrole{w}{  }\DUrole{n}{\sphinxhref{https://docs.python.org/3/library/functions.html\#int}{int}}\DUrole{w}{  }\DUrole{o}{=}\DUrole{w}{  }\DUrole{default_value}{0}}}{}
\pysigstopsignatures
\sphinxAtStartPar
Base: {\hyperref[\detokenize{estrutarq.campo:estrutarq.campo.campo_comum.CampoBasico}]{\sphinxcrossref{\sphinxcode{\sphinxupquote{estrutarq.campo.campo\_comum.CampoBasico}}}}}

\sphinxAtStartPar
Classe básica para campo inteiro
\index{bytes\_para\_valor() (método estrutarq.campo.campo\_inteiro.CampoIntBasico)@\spxentry{bytes\_para\_valor()}\spxextra{método estrutarq.campo.campo\_inteiro.CampoIntBasico}}

\begin{fulllineitems}
\phantomsection\label{\detokenize{estrutarq.campo:estrutarq.campo.campo_inteiro.CampoIntBasico.bytes_para_valor}}
\pysigstartsignatures
\pysiglinewithargsret{\sphinxbfcode{\sphinxupquote{bytes\_para\_valor}}}{\emph{\DUrole{n}{dado}\DUrole{p}{:}\DUrole{w}{  }\DUrole{n}{\sphinxhref{https://docs.python.org/3/library/stdtypes.html\#bytes}{bytes}}}}{}
\pysigstopsignatures
\sphinxAtStartPar
Conversão de uma sequência de bytes (representação textual)
para inteiro
:param dado: sequência de bytes

\end{fulllineitems}

\index{valor (propriedade estrutarq.campo.campo\_inteiro.CampoIntBasico )@\spxentry{valor}\spxextra{propriedade estrutarq.campo.campo\_inteiro.CampoIntBasico }}

\begin{fulllineitems}
\phantomsection\label{\detokenize{estrutarq.campo:estrutarq.campo.campo_inteiro.CampoIntBasico.valor}}
\pysigstartsignatures
\pysigline{\sphinxbfcode{\sphinxupquote{property\DUrole{w}{  }}}\sphinxbfcode{\sphinxupquote{valor}}\sphinxbfcode{\sphinxupquote{\DUrole{p}{:}\DUrole{w}{  }\sphinxhref{https://docs.python.org/3/library/functions.html\#int}{int}}}}
\pysigstopsignatures
\sphinxAtStartPar
Recuperação, com as devidas conversões, do atributo \sphinxcode{\sphinxupquote{\_\_valor}}
:return: o valor de \sphinxcode{\sphinxupquote{\_\_valor}}

\end{fulllineitems}

\index{valor\_para\_bytes() (método estrutarq.campo.campo\_inteiro.CampoIntBasico)@\spxentry{valor\_para\_bytes()}\spxextra{método estrutarq.campo.campo\_inteiro.CampoIntBasico}}

\begin{fulllineitems}
\phantomsection\label{\detokenize{estrutarq.campo:estrutarq.campo.campo_inteiro.CampoIntBasico.valor_para_bytes}}
\pysigstartsignatures
\pysiglinewithargsret{\sphinxbfcode{\sphinxupquote{valor\_para\_bytes}}}{}{{ $\rightarrow$ \sphinxhref{https://docs.python.org/3/library/stdtypes.html\#bytes}{bytes}}}
\pysigstopsignatures
\sphinxAtStartPar
Conversão do valor inteiro para sequência de bytes usando
representação textual e codificação UTF\sphinxhyphen{}8
:return: sequência de bytes

\end{fulllineitems}


\end{fulllineitems}

\index{CampoIntBinario (classe em estrutarq.campo.campo\_inteiro)@\spxentry{CampoIntBinario}\spxextra{classe em estrutarq.campo.campo\_inteiro}}

\begin{fulllineitems}
\phantomsection\label{\detokenize{estrutarq.campo:estrutarq.campo.campo_inteiro.CampoIntBinario}}
\pysigstartsignatures
\pysiglinewithargsret{\sphinxbfcode{\sphinxupquote{class\DUrole{w}{  }}}\sphinxcode{\sphinxupquote{estrutarq.campo.campo\_inteiro.}}\sphinxbfcode{\sphinxupquote{CampoIntBinario}}}{\emph{\DUrole{o}{**}\DUrole{n}{kwargs}}}{}
\pysigstopsignatures
\sphinxAtStartPar
Base: {\hyperref[\detokenize{estrutarq.dado:estrutarq.dado.DadoBinario}]{\sphinxcrossref{\sphinxcode{\sphinxupquote{estrutarq.dado.dado\_comum.DadoBinario}}}}}, {\hyperref[\detokenize{estrutarq.campo:estrutarq.campo.campo_inteiro.CampoIntBasico}]{\sphinxcrossref{\sphinxcode{\sphinxupquote{estrutarq.campo.campo\_inteiro.CampoIntBasico}}}}}

\sphinxAtStartPar
Classe para inteiro em formato binário (big endian) com 8 bytes
e complemento para 2 para valores negativos
\index{bytes\_para\_valor() (método estrutarq.campo.campo\_inteiro.CampoIntBinario)@\spxentry{bytes\_para\_valor()}\spxextra{método estrutarq.campo.campo\_inteiro.CampoIntBinario}}

\begin{fulllineitems}
\phantomsection\label{\detokenize{estrutarq.campo:estrutarq.campo.campo_inteiro.CampoIntBinario.bytes_para_valor}}
\pysigstartsignatures
\pysiglinewithargsret{\sphinxbfcode{\sphinxupquote{bytes\_para\_valor}}}{\emph{\DUrole{n}{dado}\DUrole{p}{:}\DUrole{w}{  }\DUrole{n}{\sphinxhref{https://docs.python.org/3/library/stdtypes.html\#bytes}{bytes}}}}{}
\pysigstopsignatures
\sphinxAtStartPar
Conversão de uma sequência de bytes (binária big\sphinxhyphen{}endian com sinal)
para inteiro
:param dado: sequência de bytes

\end{fulllineitems}

\index{numero\_bytes (atributo estrutarq.campo.campo\_inteiro.CampoIntBinario)@\spxentry{numero\_bytes}\spxextra{atributo estrutarq.campo.campo\_inteiro.CampoIntBinario}}

\begin{fulllineitems}
\phantomsection\label{\detokenize{estrutarq.campo:estrutarq.campo.campo_inteiro.CampoIntBinario.numero_bytes}}
\pysigstartsignatures
\pysigline{\sphinxbfcode{\sphinxupquote{numero\_bytes}}\sphinxbfcode{\sphinxupquote{\DUrole{w}{  }\DUrole{p}{=}\DUrole{w}{  }8}}}
\pysigstopsignatures
\end{fulllineitems}

\index{valor\_para\_bytes() (método estrutarq.campo.campo\_inteiro.CampoIntBinario)@\spxentry{valor\_para\_bytes()}\spxextra{método estrutarq.campo.campo\_inteiro.CampoIntBinario}}

\begin{fulllineitems}
\phantomsection\label{\detokenize{estrutarq.campo:estrutarq.campo.campo_inteiro.CampoIntBinario.valor_para_bytes}}
\pysigstartsignatures
\pysiglinewithargsret{\sphinxbfcode{\sphinxupquote{valor\_para\_bytes}}}{}{{ $\rightarrow$ \sphinxhref{https://docs.python.org/3/library/stdtypes.html\#bytes}{bytes}}}
\pysigstopsignatures
\sphinxAtStartPar
Conversão do valor inteiro para sequência de bytes usando
representação binária big\sphinxhyphen{}endian com sinal
:return: sequência de bytes

\end{fulllineitems}


\end{fulllineitems}

\index{CampoIntFixo (classe em estrutarq.campo.campo\_inteiro)@\spxentry{CampoIntFixo}\spxextra{classe em estrutarq.campo.campo\_inteiro}}

\begin{fulllineitems}
\phantomsection\label{\detokenize{estrutarq.campo:estrutarq.campo.campo_inteiro.CampoIntFixo}}
\pysigstartsignatures
\pysiglinewithargsret{\sphinxbfcode{\sphinxupquote{class\DUrole{w}{  }}}\sphinxcode{\sphinxupquote{estrutarq.campo.campo\_inteiro.}}\sphinxbfcode{\sphinxupquote{CampoIntFixo}}}{\emph{\DUrole{n}{comprimento}\DUrole{p}{:}\DUrole{w}{  }\DUrole{n}{\sphinxhref{https://docs.python.org/3/library/functions.html\#int}{int}}}, \emph{\DUrole{o}{**}\DUrole{n}{kwargs}}}{}
\pysigstopsignatures
\sphinxAtStartPar
Base: {\hyperref[\detokenize{estrutarq.dado:estrutarq.dado.DadoFixo}]{\sphinxcrossref{\sphinxcode{\sphinxupquote{estrutarq.dado.dado\_comum.DadoFixo}}}}}, {\hyperref[\detokenize{estrutarq.campo:estrutarq.campo.campo_inteiro.CampoIntBasico}]{\sphinxcrossref{\sphinxcode{\sphinxupquote{estrutarq.campo.campo\_inteiro.CampoIntBasico}}}}}

\sphinxAtStartPar
Classe para inteiro textual com tamanho fixo

\end{fulllineitems}

\index{CampoIntPrefixado (classe em estrutarq.campo.campo\_inteiro)@\spxentry{CampoIntPrefixado}\spxextra{classe em estrutarq.campo.campo\_inteiro}}

\begin{fulllineitems}
\phantomsection\label{\detokenize{estrutarq.campo:estrutarq.campo.campo_inteiro.CampoIntPrefixado}}
\pysigstartsignatures
\pysiglinewithargsret{\sphinxbfcode{\sphinxupquote{class\DUrole{w}{  }}}\sphinxcode{\sphinxupquote{estrutarq.campo.campo\_inteiro.}}\sphinxbfcode{\sphinxupquote{CampoIntPrefixado}}}{\emph{\DUrole{o}{**}\DUrole{n}{kwargs}}}{}
\pysigstopsignatures
\sphinxAtStartPar
Base: {\hyperref[\detokenize{estrutarq.dado:estrutarq.dado.DadoPrefixado}]{\sphinxcrossref{\sphinxcode{\sphinxupquote{estrutarq.dado.dado\_comum.DadoPrefixado}}}}}, {\hyperref[\detokenize{estrutarq.campo:estrutarq.campo.campo_inteiro.CampoIntBasico}]{\sphinxcrossref{\sphinxcode{\sphinxupquote{estrutarq.campo.campo\_inteiro.CampoIntBasico}}}}}

\sphinxAtStartPar
Classe para inteiro textual com prefixo de comprimento\_bloco

\end{fulllineitems}

\index{CampoIntTerminador (classe em estrutarq.campo.campo\_inteiro)@\spxentry{CampoIntTerminador}\spxextra{classe em estrutarq.campo.campo\_inteiro}}

\begin{fulllineitems}
\phantomsection\label{\detokenize{estrutarq.campo:estrutarq.campo.campo_inteiro.CampoIntTerminador}}
\pysigstartsignatures
\pysiglinewithargsret{\sphinxbfcode{\sphinxupquote{class\DUrole{w}{  }}}\sphinxcode{\sphinxupquote{estrutarq.campo.campo\_inteiro.}}\sphinxbfcode{\sphinxupquote{CampoIntTerminador}}}{\emph{\DUrole{n}{terminador}\DUrole{p}{:}\DUrole{w}{  }\DUrole{n}{\sphinxhref{https://docs.python.org/3/library/stdtypes.html\#bytes}{bytes}}\DUrole{w}{  }\DUrole{o}{=}\DUrole{w}{  }\DUrole{default_value}{b\textquotesingle{}\textbackslash{}x00\textquotesingle{}}}, \emph{\DUrole{o}{**}\DUrole{n}{kwargs}}}{}
\pysigstopsignatures
\sphinxAtStartPar
Base: {\hyperref[\detokenize{estrutarq.dado:estrutarq.dado.DadoTerminador}]{\sphinxcrossref{\sphinxcode{\sphinxupquote{estrutarq.dado.dado\_comum.DadoTerminador}}}}}, {\hyperref[\detokenize{estrutarq.campo:estrutarq.campo.campo_inteiro.CampoIntBasico}]{\sphinxcrossref{\sphinxcode{\sphinxupquote{estrutarq.campo.campo\_inteiro.CampoIntBasico}}}}}

\sphinxAtStartPar
Classe para inteiro textual com terminador

\end{fulllineitems}



\subsubsection{estrutarq.campo.campo\_real module}
\label{\detokenize{estrutarq.campo:module-estrutarq.campo.campo_real}}\label{\detokenize{estrutarq.campo:estrutarq-campo-campo-real-module}}\index{módulo@\spxentry{módulo}!estrutarq.campo.campo\_real@\spxentry{estrutarq.campo.campo\_real}}\index{estrutarq.campo.campo\_real@\spxentry{estrutarq.campo.campo\_real}!módulo@\spxentry{módulo}}\index{CampoRealBasico (classe em estrutarq.campo.campo\_real)@\spxentry{CampoRealBasico}\spxextra{classe em estrutarq.campo.campo\_real}}

\begin{fulllineitems}
\phantomsection\label{\detokenize{estrutarq.campo:estrutarq.campo.campo_real.CampoRealBasico}}
\pysigstartsignatures
\pysiglinewithargsret{\sphinxbfcode{\sphinxupquote{class\DUrole{w}{  }}}\sphinxcode{\sphinxupquote{estrutarq.campo.campo\_real.}}\sphinxbfcode{\sphinxupquote{CampoRealBasico}}}{\emph{\DUrole{n}{tipo}\DUrole{p}{:}\DUrole{w}{  }\DUrole{n}{\sphinxhref{https://docs.python.org/3/library/stdtypes.html\#str}{str}}}, \emph{\DUrole{n}{valor}\DUrole{p}{:}\DUrole{w}{  }\DUrole{n}{\sphinxhref{https://docs.python.org/3/library/functions.html\#float}{float}}\DUrole{w}{  }\DUrole{o}{=}\DUrole{w}{  }\DUrole{default_value}{0}}}{}
\pysigstopsignatures
\sphinxAtStartPar
Base: {\hyperref[\detokenize{estrutarq.campo:estrutarq.campo.campo_comum.CampoBasico}]{\sphinxcrossref{\sphinxcode{\sphinxupquote{estrutarq.campo.campo\_comum.CampoBasico}}}}}

\sphinxAtStartPar
Classe básica para campo real
\index{bytes\_para\_valor() (método estrutarq.campo.campo\_real.CampoRealBasico)@\spxentry{bytes\_para\_valor()}\spxextra{método estrutarq.campo.campo\_real.CampoRealBasico}}

\begin{fulllineitems}
\phantomsection\label{\detokenize{estrutarq.campo:estrutarq.campo.campo_real.CampoRealBasico.bytes_para_valor}}
\pysigstartsignatures
\pysiglinewithargsret{\sphinxbfcode{\sphinxupquote{bytes\_para\_valor}}}{\emph{\DUrole{n}{dado}\DUrole{p}{:}\DUrole{w}{  }\DUrole{n}{\sphinxhref{https://docs.python.org/3/library/stdtypes.html\#bytes}{bytes}}}}{}
\pysigstopsignatures
\sphinxAtStartPar
Conversão de sequência de bytes com valor textual para valor real
:param dado: sequência de 8 bytes

\end{fulllineitems}

\index{valor (propriedade estrutarq.campo.campo\_real.CampoRealBasico )@\spxentry{valor}\spxextra{propriedade estrutarq.campo.campo\_real.CampoRealBasico }}

\begin{fulllineitems}
\phantomsection\label{\detokenize{estrutarq.campo:estrutarq.campo.campo_real.CampoRealBasico.valor}}
\pysigstartsignatures
\pysigline{\sphinxbfcode{\sphinxupquote{property\DUrole{w}{  }}}\sphinxbfcode{\sphinxupquote{valor}}\sphinxbfcode{\sphinxupquote{\DUrole{p}{:}\DUrole{w}{  }\sphinxhref{https://docs.python.org/3/library/functions.html\#float}{float}}}}
\pysigstopsignatures
\sphinxAtStartPar
Recuperação, com as devidas conversões, do atributo \sphinxcode{\sphinxupquote{\_\_valor}}
:return: o valor de \sphinxcode{\sphinxupquote{\_\_valor}}

\end{fulllineitems}

\index{valor\_para\_bytes() (método estrutarq.campo.campo\_real.CampoRealBasico)@\spxentry{valor\_para\_bytes()}\spxextra{método estrutarq.campo.campo\_real.CampoRealBasico}}

\begin{fulllineitems}
\phantomsection\label{\detokenize{estrutarq.campo:estrutarq.campo.campo_real.CampoRealBasico.valor_para_bytes}}
\pysigstartsignatures
\pysiglinewithargsret{\sphinxbfcode{\sphinxupquote{valor\_para\_bytes}}}{}{{ $\rightarrow$ \sphinxhref{https://docs.python.org/3/library/stdtypes.html\#bytes}{bytes}}}
\pysigstopsignatures
\sphinxAtStartPar
Conversão do valor do campo para sequência de bytes textual
:return: a sequência de bytes no padrão especificado

\end{fulllineitems}


\end{fulllineitems}

\index{CampoRealBinario (classe em estrutarq.campo.campo\_real)@\spxentry{CampoRealBinario}\spxextra{classe em estrutarq.campo.campo\_real}}

\begin{fulllineitems}
\phantomsection\label{\detokenize{estrutarq.campo:estrutarq.campo.campo_real.CampoRealBinario}}
\pysigstartsignatures
\pysiglinewithargsret{\sphinxbfcode{\sphinxupquote{class\DUrole{w}{  }}}\sphinxcode{\sphinxupquote{estrutarq.campo.campo\_real.}}\sphinxbfcode{\sphinxupquote{CampoRealBinario}}}{\emph{\DUrole{o}{**}\DUrole{n}{kwargs}}}{}
\pysigstopsignatures
\sphinxAtStartPar
Base: {\hyperref[\detokenize{estrutarq.dado:estrutarq.dado.DadoBinario}]{\sphinxcrossref{\sphinxcode{\sphinxupquote{estrutarq.dado.dado\_comum.DadoBinario}}}}}, {\hyperref[\detokenize{estrutarq.campo:estrutarq.campo.campo_real.CampoRealBasico}]{\sphinxcrossref{\sphinxcode{\sphinxupquote{estrutarq.campo.campo\_real.CampoRealBasico}}}}}

\sphinxAtStartPar
Classe para real em formato binário usando IEEE 754 de precisão dupla
\index{bytes\_para\_valor() (método estrutarq.campo.campo\_real.CampoRealBinario)@\spxentry{bytes\_para\_valor()}\spxextra{método estrutarq.campo.campo\_real.CampoRealBinario}}

\begin{fulllineitems}
\phantomsection\label{\detokenize{estrutarq.campo:estrutarq.campo.campo_real.CampoRealBinario.bytes_para_valor}}
\pysigstartsignatures
\pysiglinewithargsret{\sphinxbfcode{\sphinxupquote{bytes\_para\_valor}}}{\emph{\DUrole{n}{dado}\DUrole{p}{:}\DUrole{w}{  }\DUrole{n}{\sphinxhref{https://docs.python.org/3/library/stdtypes.html\#bytes}{bytes}}}}{}
\pysigstopsignatures
\sphinxAtStartPar
Conversão de sequência de bytes com representação IEEE 754 de
precisão dupla para real
:param dado: sequência de 8 bytes

\end{fulllineitems}

\index{valor\_para\_bytes() (método estrutarq.campo.campo\_real.CampoRealBinario)@\spxentry{valor\_para\_bytes()}\spxextra{método estrutarq.campo.campo\_real.CampoRealBinario}}

\begin{fulllineitems}
\phantomsection\label{\detokenize{estrutarq.campo:estrutarq.campo.campo_real.CampoRealBinario.valor_para_bytes}}
\pysigstartsignatures
\pysiglinewithargsret{\sphinxbfcode{\sphinxupquote{valor\_para\_bytes}}}{}{{ $\rightarrow$ \sphinxhref{https://docs.python.org/3/library/stdtypes.html\#bytes}{bytes}}}
\pysigstopsignatures
\sphinxAtStartPar
Conversão do valor do campo para sequência de bytes no padrão
IEEE 754 de precisão dupla
:return: a sequência de bytes no padrão especificado

\end{fulllineitems}


\end{fulllineitems}

\index{CampoRealFixo (classe em estrutarq.campo.campo\_real)@\spxentry{CampoRealFixo}\spxextra{classe em estrutarq.campo.campo\_real}}

\begin{fulllineitems}
\phantomsection\label{\detokenize{estrutarq.campo:estrutarq.campo.campo_real.CampoRealFixo}}
\pysigstartsignatures
\pysiglinewithargsret{\sphinxbfcode{\sphinxupquote{class\DUrole{w}{  }}}\sphinxcode{\sphinxupquote{estrutarq.campo.campo\_real.}}\sphinxbfcode{\sphinxupquote{CampoRealFixo}}}{\emph{\DUrole{n}{comprimento}\DUrole{p}{:}\DUrole{w}{  }\DUrole{n}{\sphinxhref{https://docs.python.org/3/library/functions.html\#int}{int}}}, \emph{\DUrole{o}{**}\DUrole{n}{kwargs}}}{}
\pysigstopsignatures
\sphinxAtStartPar
Base: {\hyperref[\detokenize{estrutarq.dado:estrutarq.dado.DadoFixo}]{\sphinxcrossref{\sphinxcode{\sphinxupquote{estrutarq.dado.dado\_comum.DadoFixo}}}}}, {\hyperref[\detokenize{estrutarq.campo:estrutarq.campo.campo_real.CampoRealBasico}]{\sphinxcrossref{\sphinxcode{\sphinxupquote{estrutarq.campo.campo\_real.CampoRealBasico}}}}}

\sphinxAtStartPar
Classe para campo real com representação textual de comprimento\_bloco fixo

\end{fulllineitems}

\index{CampoRealPrefixado (classe em estrutarq.campo.campo\_real)@\spxentry{CampoRealPrefixado}\spxextra{classe em estrutarq.campo.campo\_real}}

\begin{fulllineitems}
\phantomsection\label{\detokenize{estrutarq.campo:estrutarq.campo.campo_real.CampoRealPrefixado}}
\pysigstartsignatures
\pysiglinewithargsret{\sphinxbfcode{\sphinxupquote{class\DUrole{w}{  }}}\sphinxcode{\sphinxupquote{estrutarq.campo.campo\_real.}}\sphinxbfcode{\sphinxupquote{CampoRealPrefixado}}}{\emph{\DUrole{o}{**}\DUrole{n}{kwargs}}}{}
\pysigstopsignatures
\sphinxAtStartPar
Base: {\hyperref[\detokenize{estrutarq.dado:estrutarq.dado.DadoPrefixado}]{\sphinxcrossref{\sphinxcode{\sphinxupquote{estrutarq.dado.dado\_comum.DadoPrefixado}}}}}, {\hyperref[\detokenize{estrutarq.campo:estrutarq.campo.campo_real.CampoRealBasico}]{\sphinxcrossref{\sphinxcode{\sphinxupquote{estrutarq.campo.campo\_real.CampoRealBasico}}}}}

\sphinxAtStartPar
Classe para campo real com representação textual de comprimento\_bloco fixo

\end{fulllineitems}

\index{CampoRealTerminador (classe em estrutarq.campo.campo\_real)@\spxentry{CampoRealTerminador}\spxextra{classe em estrutarq.campo.campo\_real}}

\begin{fulllineitems}
\phantomsection\label{\detokenize{estrutarq.campo:estrutarq.campo.campo_real.CampoRealTerminador}}
\pysigstartsignatures
\pysiglinewithargsret{\sphinxbfcode{\sphinxupquote{class\DUrole{w}{  }}}\sphinxcode{\sphinxupquote{estrutarq.campo.campo\_real.}}\sphinxbfcode{\sphinxupquote{CampoRealTerminador}}}{\emph{\DUrole{n}{terminador}\DUrole{p}{:}\DUrole{w}{  }\DUrole{n}{\sphinxhref{https://docs.python.org/3/library/stdtypes.html\#bytes}{bytes}}\DUrole{w}{  }\DUrole{o}{=}\DUrole{w}{  }\DUrole{default_value}{b\textquotesingle{}\textbackslash{}x00\textquotesingle{}}}, \emph{\DUrole{o}{**}\DUrole{n}{kwargs}}}{}
\pysigstopsignatures
\sphinxAtStartPar
Base: {\hyperref[\detokenize{estrutarq.dado:estrutarq.dado.DadoTerminador}]{\sphinxcrossref{\sphinxcode{\sphinxupquote{estrutarq.dado.dado\_comum.DadoTerminador}}}}}, {\hyperref[\detokenize{estrutarq.campo:estrutarq.campo.campo_real.CampoRealBasico}]{\sphinxcrossref{\sphinxcode{\sphinxupquote{estrutarq.campo.campo\_real.CampoRealBasico}}}}}

\sphinxAtStartPar
Classe para campo real com representação textual de comprimento\_bloco fixo

\end{fulllineitems}



\subsubsection{estrutarq.campo.campo\_tempo module}
\label{\detokenize{estrutarq.campo:module-estrutarq.campo.campo_tempo}}\label{\detokenize{estrutarq.campo:estrutarq-campo-campo-tempo-module}}\index{módulo@\spxentry{módulo}!estrutarq.campo.campo\_tempo@\spxentry{estrutarq.campo.campo\_tempo}}\index{estrutarq.campo.campo\_tempo@\spxentry{estrutarq.campo.campo\_tempo}!módulo@\spxentry{módulo}}\index{CampoDataBinario (classe em estrutarq.campo.campo\_tempo)@\spxentry{CampoDataBinario}\spxextra{classe em estrutarq.campo.campo\_tempo}}

\begin{fulllineitems}
\phantomsection\label{\detokenize{estrutarq.campo:estrutarq.campo.campo_tempo.CampoDataBinario}}
\pysigstartsignatures
\pysiglinewithargsret{\sphinxbfcode{\sphinxupquote{class\DUrole{w}{  }}}\sphinxcode{\sphinxupquote{estrutarq.campo.campo\_tempo.}}\sphinxbfcode{\sphinxupquote{CampoDataBinario}}}{\emph{\DUrole{o}{**}\DUrole{n}{kwargs}}}{}
\pysigstopsignatures
\sphinxAtStartPar
Base: {\hyperref[\detokenize{estrutarq.dado:estrutarq.dado.DadoBinario}]{\sphinxcrossref{\sphinxcode{\sphinxupquote{estrutarq.dado.dado\_comum.DadoBinario}}}}}, {\hyperref[\detokenize{estrutarq.campo:estrutarq.campo.campo_tempo.CampoTempoBasicoBinario}]{\sphinxcrossref{\sphinxcode{\sphinxupquote{estrutarq.campo.campo\_tempo.CampoTempoBasicoBinario}}}}}

\sphinxAtStartPar
Classe para armazenamento de data (dia, mês e ano) para armazenamento
em formato binário.

\end{fulllineitems}

\index{CampoDataFixo (classe em estrutarq.campo.campo\_tempo)@\spxentry{CampoDataFixo}\spxextra{classe em estrutarq.campo.campo\_tempo}}

\begin{fulllineitems}
\phantomsection\label{\detokenize{estrutarq.campo:estrutarq.campo.campo_tempo.CampoDataFixo}}
\pysigstartsignatures
\pysiglinewithargsret{\sphinxbfcode{\sphinxupquote{class\DUrole{w}{  }}}\sphinxcode{\sphinxupquote{estrutarq.campo.campo\_tempo.}}\sphinxbfcode{\sphinxupquote{CampoDataFixo}}}{\emph{\DUrole{o}{**}\DUrole{n}{kwargs}}}{}
\pysigstopsignatures
\sphinxAtStartPar
Base: {\hyperref[\detokenize{estrutarq.dado:estrutarq.dado.DadoFixo}]{\sphinxcrossref{\sphinxcode{\sphinxupquote{estrutarq.dado.dado\_comum.DadoFixo}}}}}, {\hyperref[\detokenize{estrutarq.campo:estrutarq.campo.campo_tempo.CampoTempoBasicoFixo}]{\sphinxcrossref{\sphinxcode{\sphinxupquote{estrutarq.campo.campo\_tempo.CampoTempoBasicoFixo}}}}}

\sphinxAtStartPar
Classe para data, em número de segundos desde 1/1/1970,
0h00min00s usando armazenamento em cadeia de caracteres no formato
‘formato\_data’.

\end{fulllineitems}

\index{CampoHoraBinario (classe em estrutarq.campo.campo\_tempo)@\spxentry{CampoHoraBinario}\spxextra{classe em estrutarq.campo.campo\_tempo}}

\begin{fulllineitems}
\phantomsection\label{\detokenize{estrutarq.campo:estrutarq.campo.campo_tempo.CampoHoraBinario}}
\pysigstartsignatures
\pysiglinewithargsret{\sphinxbfcode{\sphinxupquote{class\DUrole{w}{  }}}\sphinxcode{\sphinxupquote{estrutarq.campo.campo\_tempo.}}\sphinxbfcode{\sphinxupquote{CampoHoraBinario}}}{\emph{\DUrole{o}{**}\DUrole{n}{kwargs}}}{}
\pysigstopsignatures
\sphinxAtStartPar
Base: {\hyperref[\detokenize{estrutarq.dado:estrutarq.dado.DadoBinario}]{\sphinxcrossref{\sphinxcode{\sphinxupquote{estrutarq.dado.dado\_comum.DadoBinario}}}}}, {\hyperref[\detokenize{estrutarq.campo:estrutarq.campo.campo_tempo.CampoTempoBasicoBinario}]{\sphinxcrossref{\sphinxcode{\sphinxupquote{estrutarq.campo.campo\_tempo.CampoTempoBasicoBinario}}}}}

\sphinxAtStartPar
Classe para horário usando armazenamento em valor inteiro em binário,
com sinal, big\sphinxhyphen{}endian.

\end{fulllineitems}

\index{CampoHoraFixo (classe em estrutarq.campo.campo\_tempo)@\spxentry{CampoHoraFixo}\spxextra{classe em estrutarq.campo.campo\_tempo}}

\begin{fulllineitems}
\phantomsection\label{\detokenize{estrutarq.campo:estrutarq.campo.campo_tempo.CampoHoraFixo}}
\pysigstartsignatures
\pysiglinewithargsret{\sphinxbfcode{\sphinxupquote{class\DUrole{w}{  }}}\sphinxcode{\sphinxupquote{estrutarq.campo.campo\_tempo.}}\sphinxbfcode{\sphinxupquote{CampoHoraFixo}}}{\emph{\DUrole{o}{**}\DUrole{n}{kwargs}}}{}
\pysigstopsignatures
\sphinxAtStartPar
Base: {\hyperref[\detokenize{estrutarq.dado:estrutarq.dado.DadoFixo}]{\sphinxcrossref{\sphinxcode{\sphinxupquote{estrutarq.dado.dado\_comum.DadoFixo}}}}}, {\hyperref[\detokenize{estrutarq.campo:estrutarq.campo.campo_tempo.CampoTempoBasicoFixo}]{\sphinxcrossref{\sphinxcode{\sphinxupquote{estrutarq.campo.campo\_tempo.CampoTempoBasicoFixo}}}}}

\sphinxAtStartPar
Classe horário usando armazenamento em cadeia de caracteres no formato
‘formato\_hora’.

\end{fulllineitems}

\index{CampoTempoBasico (classe em estrutarq.campo.campo\_tempo)@\spxentry{CampoTempoBasico}\spxextra{classe em estrutarq.campo.campo\_tempo}}

\begin{fulllineitems}
\phantomsection\label{\detokenize{estrutarq.campo:estrutarq.campo.campo_tempo.CampoTempoBasico}}
\pysigstartsignatures
\pysiglinewithargsret{\sphinxbfcode{\sphinxupquote{class\DUrole{w}{  }}}\sphinxcode{\sphinxupquote{estrutarq.campo.campo\_tempo.}}\sphinxbfcode{\sphinxupquote{CampoTempoBasico}}}{\emph{\DUrole{n}{tipo}\DUrole{p}{:}\DUrole{w}{  }\DUrole{n}{\sphinxhref{https://docs.python.org/3/library/stdtypes.html\#str}{str}}}, \emph{\DUrole{n}{formato}\DUrole{p}{:}\DUrole{w}{  }\DUrole{n}{\sphinxhref{https://docs.python.org/3/library/stdtypes.html\#str}{str}}}, \emph{\DUrole{n}{apenas\_data}\DUrole{p}{:}\DUrole{w}{  }\DUrole{n}{\sphinxhref{https://docs.python.org/3/library/functions.html\#bool}{bool}}}, \emph{\DUrole{n}{valor}\DUrole{p}{:}\DUrole{w}{  }\DUrole{n}{\sphinxhref{https://docs.python.org/3/library/stdtypes.html\#str}{str}}\DUrole{w}{  }\DUrole{o}{=}\DUrole{w}{  }\DUrole{default_value}{\textquotesingle{}\textquotesingle{}}}, \emph{\DUrole{o}{**}\DUrole{n}{kwargs}}}{}
\pysigstopsignatures
\sphinxAtStartPar
Base: {\hyperref[\detokenize{estrutarq.campo:estrutarq.campo.campo_comum.CampoBasico}]{\sphinxcrossref{\sphinxcode{\sphinxupquote{estrutarq.campo.campo\_comum.CampoBasico}}}}}

\sphinxAtStartPar
Classe básica para campo de tempo (data + horário), armazenado
internamente como o número de segundos desde 1/1/1970, 0h00min00s.

\sphinxAtStartPar
Quando apenas a data é armazenada, o horário é ajustado para
12h00min00s, para evitar problemas com fuso horário.
\index{comprimento\_data (atributo estrutarq.campo.campo\_tempo.CampoTempoBasico)@\spxentry{comprimento\_data}\spxextra{atributo estrutarq.campo.campo\_tempo.CampoTempoBasico}}

\begin{fulllineitems}
\phantomsection\label{\detokenize{estrutarq.campo:estrutarq.campo.campo_tempo.CampoTempoBasico.comprimento_data}}
\pysigstartsignatures
\pysigline{\sphinxbfcode{\sphinxupquote{comprimento\_data}}\sphinxbfcode{\sphinxupquote{\DUrole{w}{  }\DUrole{p}{=}\DUrole{w}{  }10}}}
\pysigstopsignatures
\end{fulllineitems}

\index{comprimento\_hora (atributo estrutarq.campo.campo\_tempo.CampoTempoBasico)@\spxentry{comprimento\_hora}\spxextra{atributo estrutarq.campo.campo\_tempo.CampoTempoBasico}}

\begin{fulllineitems}
\phantomsection\label{\detokenize{estrutarq.campo:estrutarq.campo.campo_tempo.CampoTempoBasico.comprimento_hora}}
\pysigstartsignatures
\pysigline{\sphinxbfcode{\sphinxupquote{comprimento\_hora}}\sphinxbfcode{\sphinxupquote{\DUrole{w}{  }\DUrole{p}{=}\DUrole{w}{  }8}}}
\pysigstopsignatures
\end{fulllineitems}

\index{comprimento\_tempo (atributo estrutarq.campo.campo\_tempo.CampoTempoBasico)@\spxentry{comprimento\_tempo}\spxextra{atributo estrutarq.campo.campo\_tempo.CampoTempoBasico}}

\begin{fulllineitems}
\phantomsection\label{\detokenize{estrutarq.campo:estrutarq.campo.campo_tempo.CampoTempoBasico.comprimento_tempo}}
\pysigstartsignatures
\pysigline{\sphinxbfcode{\sphinxupquote{comprimento\_tempo}}\sphinxbfcode{\sphinxupquote{\DUrole{w}{  }\DUrole{p}{=}\DUrole{w}{  }19}}}
\pysigstopsignatures
\end{fulllineitems}

\index{formato\_data (atributo estrutarq.campo.campo\_tempo.CampoTempoBasico)@\spxentry{formato\_data}\spxextra{atributo estrutarq.campo.campo\_tempo.CampoTempoBasico}}

\begin{fulllineitems}
\phantomsection\label{\detokenize{estrutarq.campo:estrutarq.campo.campo_tempo.CampoTempoBasico.formato_data}}
\pysigstartsignatures
\pysigline{\sphinxbfcode{\sphinxupquote{formato\_data}}\sphinxbfcode{\sphinxupquote{\DUrole{w}{  }\DUrole{p}{=}\DUrole{w}{  }\textquotesingle{}\%Y\sphinxhyphen{}\%m\sphinxhyphen{}\%d\textquotesingle{}}}}
\pysigstopsignatures
\end{fulllineitems}

\index{formato\_hora (atributo estrutarq.campo.campo\_tempo.CampoTempoBasico)@\spxentry{formato\_hora}\spxextra{atributo estrutarq.campo.campo\_tempo.CampoTempoBasico}}

\begin{fulllineitems}
\phantomsection\label{\detokenize{estrutarq.campo:estrutarq.campo.campo_tempo.CampoTempoBasico.formato_hora}}
\pysigstartsignatures
\pysigline{\sphinxbfcode{\sphinxupquote{formato\_hora}}\sphinxbfcode{\sphinxupquote{\DUrole{w}{  }\DUrole{p}{=}\DUrole{w}{  }\textquotesingle{}\%H:\%M:\%S\textquotesingle{}}}}
\pysigstopsignatures
\end{fulllineitems}

\index{formato\_tempo (atributo estrutarq.campo.campo\_tempo.CampoTempoBasico)@\spxentry{formato\_tempo}\spxextra{atributo estrutarq.campo.campo\_tempo.CampoTempoBasico}}

\begin{fulllineitems}
\phantomsection\label{\detokenize{estrutarq.campo:estrutarq.campo.campo_tempo.CampoTempoBasico.formato_tempo}}
\pysigstartsignatures
\pysigline{\sphinxbfcode{\sphinxupquote{formato\_tempo}}\sphinxbfcode{\sphinxupquote{\DUrole{w}{  }\DUrole{p}{=}\DUrole{w}{  }\textquotesingle{}\%Y\sphinxhyphen{}\%m\sphinxhyphen{}\%d \%H:\%M:\%S\textquotesingle{}}}}
\pysigstopsignatures
\end{fulllineitems}

\index{segundos (propriedade estrutarq.campo.campo\_tempo.CampoTempoBasico )@\spxentry{segundos}\spxextra{propriedade estrutarq.campo.campo\_tempo.CampoTempoBasico }}

\begin{fulllineitems}
\phantomsection\label{\detokenize{estrutarq.campo:estrutarq.campo.campo_tempo.CampoTempoBasico.segundos}}
\pysigstartsignatures
\pysigline{\sphinxbfcode{\sphinxupquote{property\DUrole{w}{  }}}\sphinxbfcode{\sphinxupquote{segundos}}\sphinxbfcode{\sphinxupquote{\DUrole{p}{:}\DUrole{w}{  }\sphinxhref{https://docs.python.org/3/library/functions.html\#int}{int}}}}
\pysigstopsignatures
\end{fulllineitems}

\index{valor (propriedade estrutarq.campo.campo\_tempo.CampoTempoBasico )@\spxentry{valor}\spxextra{propriedade estrutarq.campo.campo\_tempo.CampoTempoBasico }}

\begin{fulllineitems}
\phantomsection\label{\detokenize{estrutarq.campo:estrutarq.campo.campo_tempo.CampoTempoBasico.valor}}
\pysigstartsignatures
\pysigline{\sphinxbfcode{\sphinxupquote{property\DUrole{w}{  }}}\sphinxbfcode{\sphinxupquote{valor}}\sphinxbfcode{\sphinxupquote{\DUrole{p}{:}\DUrole{w}{  }\sphinxhref{https://docs.python.org/3/library/stdtypes.html\#str}{str}}}}
\pysigstopsignatures
\sphinxAtStartPar
Recuperação, com as devidas conversões, do atributo \sphinxcode{\sphinxupquote{\_\_valor}}
:return: o valor de \sphinxcode{\sphinxupquote{\_\_valor}}

\end{fulllineitems}


\end{fulllineitems}

\index{CampoTempoBasicoBinario (classe em estrutarq.campo.campo\_tempo)@\spxentry{CampoTempoBasicoBinario}\spxextra{classe em estrutarq.campo.campo\_tempo}}

\begin{fulllineitems}
\phantomsection\label{\detokenize{estrutarq.campo:estrutarq.campo.campo_tempo.CampoTempoBasicoBinario}}
\pysigstartsignatures
\pysiglinewithargsret{\sphinxbfcode{\sphinxupquote{class\DUrole{w}{  }}}\sphinxcode{\sphinxupquote{estrutarq.campo.campo\_tempo.}}\sphinxbfcode{\sphinxupquote{CampoTempoBasicoBinario}}}{\emph{\DUrole{o}{*}\DUrole{n}{args}}, \emph{\DUrole{o}{**}\DUrole{n}{kwargs}}}{}
\pysigstopsignatures
\sphinxAtStartPar
Base: {\hyperref[\detokenize{estrutarq.campo:estrutarq.campo.campo_tempo.CampoTempoBasico}]{\sphinxcrossref{\sphinxcode{\sphinxupquote{estrutarq.campo.campo\_tempo.CampoTempoBasico}}}}}

\sphinxAtStartPar
Implementação das conversões tempo\sphinxhyphen{}\textgreater{} binário e binário\sphinxhyphen{}\textgreater{}tempo
\index{bytes\_para\_valor() (método estrutarq.campo.campo\_tempo.CampoTempoBasicoBinario)@\spxentry{bytes\_para\_valor()}\spxextra{método estrutarq.campo.campo\_tempo.CampoTempoBasicoBinario}}

\begin{fulllineitems}
\phantomsection\label{\detokenize{estrutarq.campo:estrutarq.campo.campo_tempo.CampoTempoBasicoBinario.bytes_para_valor}}
\pysigstartsignatures
\pysiglinewithargsret{\sphinxbfcode{\sphinxupquote{bytes\_para\_valor}}}{\emph{\DUrole{n}{dado}\DUrole{p}{:}\DUrole{w}{  }\DUrole{n}{\sphinxhref{https://docs.python.org/3/library/stdtypes.html\#bytes}{bytes}}}}{}
\pysigstopsignatures
\sphinxAtStartPar
Conversão da representação binária (8 bytes, big\sphinxhyphen{}endian, com sinal)
para valor inteiro de segundos
:param dado: bytes da representação do inteiro em binário

\end{fulllineitems}

\index{valor\_para\_bytes() (método estrutarq.campo.campo\_tempo.CampoTempoBasicoBinario)@\spxentry{valor\_para\_bytes()}\spxextra{método estrutarq.campo.campo\_tempo.CampoTempoBasicoBinario}}

\begin{fulllineitems}
\phantomsection\label{\detokenize{estrutarq.campo:estrutarq.campo.campo_tempo.CampoTempoBasicoBinario.valor_para_bytes}}
\pysigstartsignatures
\pysiglinewithargsret{\sphinxbfcode{\sphinxupquote{valor\_para\_bytes}}}{}{{ $\rightarrow$ \sphinxhref{https://docs.python.org/3/library/stdtypes.html\#bytes}{bytes}}}
\pysigstopsignatures
\sphinxAtStartPar
Conversão do valor do tempo em segundos para representação em
inteiro binário (8 bytes, big\sphinxhyphen{}endian, com sinal)
:return: a sequência de bytes

\end{fulllineitems}


\end{fulllineitems}

\index{CampoTempoBasicoFixo (classe em estrutarq.campo.campo\_tempo)@\spxentry{CampoTempoBasicoFixo}\spxextra{classe em estrutarq.campo.campo\_tempo}}

\begin{fulllineitems}
\phantomsection\label{\detokenize{estrutarq.campo:estrutarq.campo.campo_tempo.CampoTempoBasicoFixo}}
\pysigstartsignatures
\pysiglinewithargsret{\sphinxbfcode{\sphinxupquote{class\DUrole{w}{  }}}\sphinxcode{\sphinxupquote{estrutarq.campo.campo\_tempo.}}\sphinxbfcode{\sphinxupquote{CampoTempoBasicoFixo}}}{\emph{\DUrole{o}{*}\DUrole{n}{args}}, \emph{\DUrole{o}{**}\DUrole{n}{kwargs}}}{}
\pysigstopsignatures
\sphinxAtStartPar
Base: {\hyperref[\detokenize{estrutarq.campo:estrutarq.campo.campo_tempo.CampoTempoBasico}]{\sphinxcrossref{\sphinxcode{\sphinxupquote{estrutarq.campo.campo\_tempo.CampoTempoBasico}}}}}

\sphinxAtStartPar
Implementação das conversões tempo\sphinxhyphen{}\textgreater{} binário e binário\sphinxhyphen{}\textgreater{}tempo
\index{bytes\_para\_valor() (método estrutarq.campo.campo\_tempo.CampoTempoBasicoFixo)@\spxentry{bytes\_para\_valor()}\spxextra{método estrutarq.campo.campo\_tempo.CampoTempoBasicoFixo}}

\begin{fulllineitems}
\phantomsection\label{\detokenize{estrutarq.campo:estrutarq.campo.campo_tempo.CampoTempoBasicoFixo.bytes_para_valor}}
\pysigstartsignatures
\pysiglinewithargsret{\sphinxbfcode{\sphinxupquote{bytes\_para\_valor}}}{\emph{\DUrole{n}{dado}\DUrole{p}{:}\DUrole{w}{  }\DUrole{n}{\sphinxhref{https://docs.python.org/3/library/stdtypes.html\#bytes}{bytes}}}}{}
\pysigstopsignatures
\sphinxAtStartPar
Conversão da representação binária (8 bytes, big\sphinxhyphen{}endian, com sinal)
para valor inteiro de segundos
:param dado: bytes da representação do inteiro em binário

\end{fulllineitems}

\index{valor\_para\_bytes() (método estrutarq.campo.campo\_tempo.CampoTempoBasicoFixo)@\spxentry{valor\_para\_bytes()}\spxextra{método estrutarq.campo.campo\_tempo.CampoTempoBasicoFixo}}

\begin{fulllineitems}
\phantomsection\label{\detokenize{estrutarq.campo:estrutarq.campo.campo_tempo.CampoTempoBasicoFixo.valor_para_bytes}}
\pysigstartsignatures
\pysiglinewithargsret{\sphinxbfcode{\sphinxupquote{valor\_para\_bytes}}}{}{{ $\rightarrow$ \sphinxhref{https://docs.python.org/3/library/stdtypes.html\#bytes}{bytes}}}
\pysigstopsignatures
\sphinxAtStartPar
Conversão do valor do tempo em segundos para representação em
inteiro binário (8 bytes, big\sphinxhyphen{}endian, com sinal)
:return: a sequência de bytes

\end{fulllineitems}


\end{fulllineitems}

\index{CampoTempoBinario (classe em estrutarq.campo.campo\_tempo)@\spxentry{CampoTempoBinario}\spxextra{classe em estrutarq.campo.campo\_tempo}}

\begin{fulllineitems}
\phantomsection\label{\detokenize{estrutarq.campo:estrutarq.campo.campo_tempo.CampoTempoBinario}}
\pysigstartsignatures
\pysiglinewithargsret{\sphinxbfcode{\sphinxupquote{class\DUrole{w}{  }}}\sphinxcode{\sphinxupquote{estrutarq.campo.campo\_tempo.}}\sphinxbfcode{\sphinxupquote{CampoTempoBinario}}}{\emph{\DUrole{o}{**}\DUrole{n}{kwargs}}}{}
\pysigstopsignatures
\sphinxAtStartPar
Base: {\hyperref[\detokenize{estrutarq.dado:estrutarq.dado.DadoBinario}]{\sphinxcrossref{\sphinxcode{\sphinxupquote{estrutarq.dado.dado\_comum.DadoBinario}}}}}, {\hyperref[\detokenize{estrutarq.campo:estrutarq.campo.campo_tempo.CampoTempoBasicoBinario}]{\sphinxcrossref{\sphinxcode{\sphinxupquote{estrutarq.campo.campo\_tempo.CampoTempoBasicoBinario}}}}}

\sphinxAtStartPar
Classe para tempo (data + horário), em número de segundos desde 1/1/1970,
0h00min00s usando armazenamento em valor inteiro em binário, com sinal,
big\sphinxhyphen{}endian.

\end{fulllineitems}

\index{CampoTempoFixo (classe em estrutarq.campo.campo\_tempo)@\spxentry{CampoTempoFixo}\spxextra{classe em estrutarq.campo.campo\_tempo}}

\begin{fulllineitems}
\phantomsection\label{\detokenize{estrutarq.campo:estrutarq.campo.campo_tempo.CampoTempoFixo}}
\pysigstartsignatures
\pysiglinewithargsret{\sphinxbfcode{\sphinxupquote{class\DUrole{w}{  }}}\sphinxcode{\sphinxupquote{estrutarq.campo.campo\_tempo.}}\sphinxbfcode{\sphinxupquote{CampoTempoFixo}}}{\emph{\DUrole{o}{**}\DUrole{n}{kwargs}}}{}
\pysigstopsignatures
\sphinxAtStartPar
Base: {\hyperref[\detokenize{estrutarq.dado:estrutarq.dado.DadoFixo}]{\sphinxcrossref{\sphinxcode{\sphinxupquote{estrutarq.dado.dado\_comum.DadoFixo}}}}}, {\hyperref[\detokenize{estrutarq.campo:estrutarq.campo.campo_tempo.CampoTempoBasicoFixo}]{\sphinxcrossref{\sphinxcode{\sphinxupquote{estrutarq.campo.campo\_tempo.CampoTempoBasicoFixo}}}}}

\sphinxAtStartPar
Classe para tempo (data + horário), em número de segundos desde 1/1/1970,
0h00min00s usando armazenamento em cadeia de caracteres no formato
‘formato\_tempo’.

\end{fulllineitems}



\subsubsection{Module contents}
\label{\detokenize{estrutarq.campo:module-estrutarq.campo}}\label{\detokenize{estrutarq.campo:module-contents}}\index{módulo@\spxentry{módulo}!estrutarq.campo@\spxentry{estrutarq.campo}}\index{estrutarq.campo@\spxentry{estrutarq.campo}!módulo@\spxentry{módulo}}
\sphinxAtStartPar
Módulo: \sphinxcode{\sphinxupquote{campo}}

\sphinxAtStartPar
Implementação de representações de campos.

\sphinxstepscope


\subsection{Pacote \sphinxstyleliteralintitle{\sphinxupquote{estrutarq.registro}}}
\label{\detokenize{estrutarq.registro:pacote-estrutarq-registro}}\label{\detokenize{estrutarq.registro::doc}}

\subsubsection{Módulo \sphinxstyleliteralintitle{\sphinxupquote{estrutarq.registro.registro\_comum}}}
\label{\detokenize{estrutarq.registro:module-estrutarq.registro.registro_comum}}\label{\detokenize{estrutarq.registro:modulo-estrutarq-registro-registro-comum}}\index{módulo@\spxentry{módulo}!estrutarq.registro.registro\_comum@\spxentry{estrutarq.registro.registro\_comum}}\index{estrutarq.registro.registro\_comum@\spxentry{estrutarq.registro.registro\_comum}!módulo@\spxentry{módulo}}
\sphinxAtStartPar
Registros
\index{RegistroBasico (classe em estrutarq.registro.registro\_comum)@\spxentry{RegistroBasico}\spxextra{classe em estrutarq.registro.registro\_comum}}

\begin{fulllineitems}
\phantomsection\label{\detokenize{estrutarq.registro:estrutarq.registro.registro_comum.RegistroBasico}}
\pysigstartsignatures
\pysiglinewithargsret{\sphinxbfcode{\sphinxupquote{class\DUrole{w}{  }}}\sphinxcode{\sphinxupquote{estrutarq.registro.registro\_comum.}}\sphinxbfcode{\sphinxupquote{RegistroBasico}}}{\emph{\DUrole{n}{tipo}\DUrole{p}{:}\DUrole{w}{  }\DUrole{n}{\sphinxhref{https://docs.python.org/3/library/stdtypes.html\#str}{str}}}, \emph{\DUrole{o}{*}\DUrole{n}{lista\_campos}}}{}
\pysigstopsignatures
\sphinxAtStartPar
Base: {\hyperref[\detokenize{estrutarq.dado:estrutarq.dado.DadoBasico}]{\sphinxcrossref{\sphinxcode{\sphinxupquote{estrutarq.dado.dado\_comum.DadoBasico}}}}}

\sphinxAtStartPar
Classe básica para registros

\sphinxAtStartPar
\sphinxstylestrong{Utiliza @DynamicAttrs}
\index{adicione\_campos() (método estrutarq.registro.registro\_comum.RegistroBasico)@\spxentry{adicione\_campos()}\spxextra{método estrutarq.registro.registro\_comum.RegistroBasico}}

\begin{fulllineitems}
\phantomsection\label{\detokenize{estrutarq.registro:estrutarq.registro.registro_comum.RegistroBasico.adicione_campos}}
\pysigstartsignatures
\pysiglinewithargsret{\sphinxbfcode{\sphinxupquote{adicione\_campos}}}{\emph{\DUrole{o}{*}\DUrole{n}{lista\_campos}}}{}
\pysigstopsignatures
\sphinxAtStartPar
Inclusão de uma sequência de campos ao registro
:param lista\_campos: uma sequência de um ou mais campos, cada um
\begin{quote}

\sphinxAtStartPar
especificado pela tupla (nome\_arquivo, campo), com
nome\_arquivo (str) sendo o nome\_arquivo do campo e campo sendo
uma instância de um campo válido
\end{quote}

\end{fulllineitems}

\index{comprimento() (método estrutarq.registro.registro\_comum.RegistroBasico)@\spxentry{comprimento()}\spxextra{método estrutarq.registro.registro\_comum.RegistroBasico}}

\begin{fulllineitems}
\phantomsection\label{\detokenize{estrutarq.registro:estrutarq.registro.registro_comum.RegistroBasico.comprimento}}
\pysigstartsignatures
\pysiglinewithargsret{\sphinxbfcode{\sphinxupquote{comprimento}}}{}{}
\pysigstopsignatures
\sphinxAtStartPar
Retorna o comprimento do registro em bytes caso ele tenha comprimento
total fixo
:return: o comprimento do registro em bytes ou None se tiver
comprimento variável

\end{fulllineitems}

\index{copy() (método estrutarq.registro.registro\_comum.RegistroBasico)@\spxentry{copy()}\spxextra{método estrutarq.registro.registro\_comum.RegistroBasico}}

\begin{fulllineitems}
\phantomsection\label{\detokenize{estrutarq.registro:estrutarq.registro.registro_comum.RegistroBasico.copy}}
\pysigstartsignatures
\pysiglinewithargsret{\sphinxbfcode{\sphinxupquote{copy}}}{}{}
\pysigstopsignatures
\sphinxAtStartPar
Cópia “profunda” deste campo
:return: outra instância com os mesmos valores

\end{fulllineitems}

\index{de\_bytes() (método estrutarq.registro.registro\_comum.RegistroBasico)@\spxentry{de\_bytes()}\spxextra{método estrutarq.registro.registro\_comum.RegistroBasico}}

\begin{fulllineitems}
\phantomsection\label{\detokenize{estrutarq.registro:estrutarq.registro.registro_comum.RegistroBasico.de_bytes}}
\pysigstartsignatures
\pysiglinewithargsret{\sphinxbfcode{\sphinxupquote{de\_bytes}}}{\emph{\DUrole{n}{dados\_registro}\DUrole{p}{:}\DUrole{w}{  }\DUrole{n}{\sphinxhref{https://docs.python.org/3/library/stdtypes.html\#bytes}{bytes}}}}{}
\pysigstopsignatures
\sphinxAtStartPar
Obtenção dos bytes de cada campo a partir dos bytes do registro inteiro
:param dados\_registro: sequência de bytes do registro

\end{fulllineitems}

\index{escreva() (método estrutarq.registro.registro\_comum.RegistroBasico)@\spxentry{escreva()}\spxextra{método estrutarq.registro.registro\_comum.RegistroBasico}}

\begin{fulllineitems}
\phantomsection\label{\detokenize{estrutarq.registro:estrutarq.registro.registro_comum.RegistroBasico.escreva}}
\pysigstartsignatures
\pysiglinewithargsret{\sphinxbfcode{\sphinxupquote{escreva}}}{\emph{\DUrole{n}{arquivo}}}{}
\pysigstopsignatures
\sphinxAtStartPar
Escrita do registro no arquivo
:param arquivo:

\end{fulllineitems}

\index{leia() (método estrutarq.registro.registro\_comum.RegistroBasico)@\spxentry{leia()}\spxextra{método estrutarq.registro.registro\_comum.RegistroBasico}}

\begin{fulllineitems}
\phantomsection\label{\detokenize{estrutarq.registro:estrutarq.registro.registro_comum.RegistroBasico.leia}}
\pysigstartsignatures
\pysiglinewithargsret{\sphinxbfcode{\sphinxupquote{leia}}}{\emph{\DUrole{n}{arquivo}}}{}
\pysigstopsignatures
\sphinxAtStartPar
Obtenção de um registro a partir do arquivo
:param arquivo: arquivo binário aberto com permissão de leitura

\end{fulllineitems}

\index{para\_bytes() (método estrutarq.registro.registro\_comum.RegistroBasico)@\spxentry{para\_bytes()}\spxextra{método estrutarq.registro.registro\_comum.RegistroBasico}}

\begin{fulllineitems}
\phantomsection\label{\detokenize{estrutarq.registro:estrutarq.registro.registro_comum.RegistroBasico.para_bytes}}
\pysigstartsignatures
\pysiglinewithargsret{\sphinxbfcode{\sphinxupquote{para\_bytes}}}{}{{ $\rightarrow$ \sphinxhref{https://docs.python.org/3/library/stdtypes.html\#bytes}{bytes}}}
\pysigstopsignatures
\sphinxAtStartPar
Criação dos bytes do registro pela concatenação dos bytes dos campos,
sucessivamente
:return: sequência dos bytes dos campos

\end{fulllineitems}

\index{tem\_comprimento\_fixo() (método estrutarq.registro.registro\_comum.RegistroBasico)@\spxentry{tem\_comprimento\_fixo()}\spxextra{método estrutarq.registro.registro\_comum.RegistroBasico}}

\begin{fulllineitems}
\phantomsection\label{\detokenize{estrutarq.registro:estrutarq.registro.registro_comum.RegistroBasico.tem_comprimento_fixo}}
\pysigstartsignatures
\pysiglinewithargsret{\sphinxbfcode{\sphinxupquote{tem\_comprimento\_fixo}}}{}{}
\pysigstopsignatures
\sphinxAtStartPar
Verifica se o registro tem comprimento fixo
:return: True se o comprimento for fixo

\sphinxAtStartPar
O registro é considerado de tamanho fixo se qualquer uma das
propriedades foram verdadeiras:
\begin{enumerate}
\sphinxsetlistlabels{\arabic}{enumi}{enumii}{}{)}%
\item {} 
\sphinxAtStartPar
o registro tem é marcado com \_comprimento\_fixo == True

\item {} 
\sphinxAtStartPar
todos os campos tiverem comprimento fixo

\end{enumerate}

\end{fulllineitems}

\index{tipo (propriedade estrutarq.registro.registro\_comum.RegistroBasico )@\spxentry{tipo}\spxextra{propriedade estrutarq.registro.registro\_comum.RegistroBasico }}

\begin{fulllineitems}
\phantomsection\label{\detokenize{estrutarq.registro:estrutarq.registro.registro_comum.RegistroBasico.tipo}}
\pysigstartsignatures
\pysigline{\sphinxbfcode{\sphinxupquote{property\DUrole{w}{  }}}\sphinxbfcode{\sphinxupquote{tipo}}}
\pysigstopsignatures
\end{fulllineitems}


\end{fulllineitems}

\index{RegistroBruto (classe em estrutarq.registro.registro\_comum)@\spxentry{RegistroBruto}\spxextra{classe em estrutarq.registro.registro\_comum}}

\begin{fulllineitems}
\phantomsection\label{\detokenize{estrutarq.registro:estrutarq.registro.registro_comum.RegistroBruto}}
\pysigstartsignatures
\pysiglinewithargsret{\sphinxbfcode{\sphinxupquote{class\DUrole{w}{  }}}\sphinxcode{\sphinxupquote{estrutarq.registro.registro\_comum.}}\sphinxbfcode{\sphinxupquote{RegistroBruto}}}{\emph{\DUrole{o}{*}\DUrole{n}{lista\_campos}}}{}
\pysigstopsignatures
\sphinxAtStartPar
Base: {\hyperref[\detokenize{estrutarq.dado:estrutarq.dado.DadoBruto}]{\sphinxcrossref{\sphinxcode{\sphinxupquote{estrutarq.dado.dado\_comum.DadoBruto}}}}}, {\hyperref[\detokenize{estrutarq.registro:estrutarq.registro.registro_comum.RegistroBasico}]{\sphinxcrossref{\sphinxcode{\sphinxupquote{estrutarq.registro.registro\_comum.RegistroBasico}}}}}

\sphinxAtStartPar
Classe básica para registro, com controle exclusivamente pelo número
de campos

\sphinxAtStartPar
Utiliza @DynamicAttrs

\end{fulllineitems}

\index{RegistroFixo (classe em estrutarq.registro.registro\_comum)@\spxentry{RegistroFixo}\spxextra{classe em estrutarq.registro.registro\_comum}}

\begin{fulllineitems}
\phantomsection\label{\detokenize{estrutarq.registro:estrutarq.registro.registro_comum.RegistroFixo}}
\pysigstartsignatures
\pysiglinewithargsret{\sphinxbfcode{\sphinxupquote{class\DUrole{w}{  }}}\sphinxcode{\sphinxupquote{estrutarq.registro.registro\_comum.}}\sphinxbfcode{\sphinxupquote{RegistroFixo}}}{\emph{\DUrole{n}{comprimento}\DUrole{p}{:}\DUrole{w}{  }\DUrole{n}{\sphinxhref{https://docs.python.org/3/library/functions.html\#int}{int}}}, \emph{\DUrole{o}{*}\DUrole{n}{lista\_campos}}}{}
\pysigstopsignatures
\sphinxAtStartPar
Base: {\hyperref[\detokenize{estrutarq.dado:estrutarq.dado.DadoFixo}]{\sphinxcrossref{\sphinxcode{\sphinxupquote{estrutarq.dado.dado\_comum.DadoFixo}}}}}, {\hyperref[\detokenize{estrutarq.registro:estrutarq.registro.registro_comum.RegistroBasico}]{\sphinxcrossref{\sphinxcode{\sphinxupquote{estrutarq.registro.registro\_comum.RegistroBasico}}}}}

\sphinxAtStartPar
Classe para registros com terminador

\end{fulllineitems}

\index{RegistroPrefixado (classe em estrutarq.registro.registro\_comum)@\spxentry{RegistroPrefixado}\spxextra{classe em estrutarq.registro.registro\_comum}}

\begin{fulllineitems}
\phantomsection\label{\detokenize{estrutarq.registro:estrutarq.registro.registro_comum.RegistroPrefixado}}
\pysigstartsignatures
\pysiglinewithargsret{\sphinxbfcode{\sphinxupquote{class\DUrole{w}{  }}}\sphinxcode{\sphinxupquote{estrutarq.registro.registro\_comum.}}\sphinxbfcode{\sphinxupquote{RegistroPrefixado}}}{\emph{\DUrole{o}{*}\DUrole{n}{lista\_campos}}}{}
\pysigstopsignatures
\sphinxAtStartPar
Base: {\hyperref[\detokenize{estrutarq.dado:estrutarq.dado.DadoPrefixado}]{\sphinxcrossref{\sphinxcode{\sphinxupquote{estrutarq.dado.dado\_comum.DadoPrefixado}}}}}, {\hyperref[\detokenize{estrutarq.registro:estrutarq.registro.registro_comum.RegistroBasico}]{\sphinxcrossref{\sphinxcode{\sphinxupquote{estrutarq.registro.registro\_comum.RegistroBasico}}}}}

\sphinxAtStartPar
Classe para registros prefixados pelo comprimento

\end{fulllineitems}

\index{RegistroTerminador (classe em estrutarq.registro.registro\_comum)@\spxentry{RegistroTerminador}\spxextra{classe em estrutarq.registro.registro\_comum}}

\begin{fulllineitems}
\phantomsection\label{\detokenize{estrutarq.registro:estrutarq.registro.registro_comum.RegistroTerminador}}
\pysigstartsignatures
\pysiglinewithargsret{\sphinxbfcode{\sphinxupquote{class\DUrole{w}{  }}}\sphinxcode{\sphinxupquote{estrutarq.registro.registro\_comum.}}\sphinxbfcode{\sphinxupquote{RegistroTerminador}}}{\emph{\DUrole{o}{*}\DUrole{n}{lista\_campos}}}{}
\pysigstopsignatures
\sphinxAtStartPar
Base: {\hyperref[\detokenize{estrutarq.dado:estrutarq.dado.DadoTerminador}]{\sphinxcrossref{\sphinxcode{\sphinxupquote{estrutarq.dado.dado\_comum.DadoTerminador}}}}}, {\hyperref[\detokenize{estrutarq.registro:estrutarq.registro.registro_comum.RegistroBasico}]{\sphinxcrossref{\sphinxcode{\sphinxupquote{estrutarq.registro.registro\_comum.RegistroBasico}}}}}

\sphinxAtStartPar
Classe para registros com terminador

\end{fulllineitems}


\sphinxstepscope


\subsection{estrutarq.arquivo package}
\label{\detokenize{estrutarq.arquivo:estrutarq-arquivo-package}}\label{\detokenize{estrutarq.arquivo::doc}}

\subsubsection{Submodules}
\label{\detokenize{estrutarq.arquivo:submodules}}

\subsubsection{estrutarq.arquivo.arquivo\_comum module}
\label{\detokenize{estrutarq.arquivo:module-estrutarq.arquivo.arquivo_comum}}\label{\detokenize{estrutarq.arquivo:estrutarq-arquivo-arquivo-comum-module}}\index{módulo@\spxentry{módulo}!estrutarq.arquivo.arquivo\_comum@\spxentry{estrutarq.arquivo.arquivo\_comum}}\index{estrutarq.arquivo.arquivo\_comum@\spxentry{estrutarq.arquivo.arquivo\_comum}!módulo@\spxentry{módulo}}\index{ArquivoBasico (classe em estrutarq.arquivo.arquivo\_comum)@\spxentry{ArquivoBasico}\spxextra{classe em estrutarq.arquivo.arquivo\_comum}}

\begin{fulllineitems}
\phantomsection\label{\detokenize{estrutarq.arquivo:estrutarq.arquivo.arquivo_comum.ArquivoBasico}}
\pysigstartsignatures
\pysiglinewithargsret{\sphinxbfcode{\sphinxupquote{class\DUrole{w}{  }}}\sphinxcode{\sphinxupquote{estrutarq.arquivo.arquivo\_comum.}}\sphinxbfcode{\sphinxupquote{ArquivoBasico}}}{\emph{\DUrole{n}{nome\_arquivo}\DUrole{p}{:}\DUrole{w}{  }\DUrole{n}{\sphinxhref{https://docs.python.org/3/library/stdtypes.html\#str}{str}}}, \emph{\DUrole{n}{tipo}\DUrole{p}{:}\DUrole{w}{  }\DUrole{n}{\sphinxhref{https://docs.python.org/3/library/stdtypes.html\#str}{str}}}, \emph{\DUrole{n}{novo}\DUrole{p}{:}\DUrole{w}{  }\DUrole{n}{\sphinxhref{https://docs.python.org/3/library/functions.html\#bool}{bool}}\DUrole{w}{  }\DUrole{o}{=}\DUrole{w}{  }\DUrole{default_value}{False}}}{}
\pysigstopsignatures
\sphinxAtStartPar
Base: \sphinxhref{https://docs.python.org/3/library/functions.html\#object}{\sphinxcode{\sphinxupquote{object}}}

\sphinxAtStartPar
Gerenciador dedicado a um único arquivo aberto
\index{escreva() (método estrutarq.arquivo.arquivo\_comum.ArquivoBasico)@\spxentry{escreva()}\spxextra{método estrutarq.arquivo.arquivo\_comum.ArquivoBasico}}

\begin{fulllineitems}
\phantomsection\label{\detokenize{estrutarq.arquivo:estrutarq.arquivo.arquivo_comum.ArquivoBasico.escreva}}
\pysigstartsignatures
\pysiglinewithargsret{\sphinxbfcode{\sphinxupquote{abstract\DUrole{w}{  }}}\sphinxbfcode{\sphinxupquote{escreva}}}{\emph{\DUrole{n}{registro}\DUrole{p}{:}\DUrole{w}{  }\DUrole{n}{{\hyperref[\detokenize{estrutarq.registro:estrutarq.registro.registro_comum.RegistroBasico}]{\sphinxcrossref{estrutarq.registro.registro\_comum.RegistroBasico}}}}}}{}
\pysigstopsignatures
\sphinxAtStartPar
Gravação de um registro no arquivo

\end{fulllineitems}

\index{feche() (método estrutarq.arquivo.arquivo\_comum.ArquivoBasico)@\spxentry{feche()}\spxextra{método estrutarq.arquivo.arquivo\_comum.ArquivoBasico}}

\begin{fulllineitems}
\phantomsection\label{\detokenize{estrutarq.arquivo:estrutarq.arquivo.arquivo_comum.ArquivoBasico.feche}}
\pysigstartsignatures
\pysiglinewithargsret{\sphinxbfcode{\sphinxupquote{feche}}}{}{}
\pysigstopsignatures
\sphinxAtStartPar
Fechamento do arquivo associado

\end{fulllineitems}

\index{leia() (método estrutarq.arquivo.arquivo\_comum.ArquivoBasico)@\spxentry{leia()}\spxextra{método estrutarq.arquivo.arquivo\_comum.ArquivoBasico}}

\begin{fulllineitems}
\phantomsection\label{\detokenize{estrutarq.arquivo:estrutarq.arquivo.arquivo_comum.ArquivoBasico.leia}}
\pysigstartsignatures
\pysiglinewithargsret{\sphinxbfcode{\sphinxupquote{abstract\DUrole{w}{  }}}\sphinxbfcode{\sphinxupquote{leia}}}{}{{ $\rightarrow$ {\hyperref[\detokenize{estrutarq.registro:estrutarq.registro.registro_comum.RegistroBasico}]{\sphinxcrossref{estrutarq.registro.registro\_comum.RegistroBasico}}}}}
\pysigstopsignatures
\sphinxAtStartPar
Leitura de um registro do arquivo
:return: o registro lido

\end{fulllineitems}

\index{posicao\_atual() (método estrutarq.arquivo.arquivo\_comum.ArquivoBasico)@\spxentry{posicao\_atual()}\spxextra{método estrutarq.arquivo.arquivo\_comum.ArquivoBasico}}

\begin{fulllineitems}
\phantomsection\label{\detokenize{estrutarq.arquivo:estrutarq.arquivo.arquivo_comum.ArquivoBasico.posicao_atual}}
\pysigstartsignatures
\pysiglinewithargsret{\sphinxbfcode{\sphinxupquote{posicao\_atual}}}{}{}
\pysigstopsignatures
\sphinxAtStartPar
Posição atual do arquivo
:return:

\end{fulllineitems}


\end{fulllineitems}

\index{ArquivoSimples (classe em estrutarq.arquivo.arquivo\_comum)@\spxentry{ArquivoSimples}\spxextra{classe em estrutarq.arquivo.arquivo\_comum}}

\begin{fulllineitems}
\phantomsection\label{\detokenize{estrutarq.arquivo:estrutarq.arquivo.arquivo_comum.ArquivoSimples}}
\pysigstartsignatures
\pysiglinewithargsret{\sphinxbfcode{\sphinxupquote{class\DUrole{w}{  }}}\sphinxcode{\sphinxupquote{estrutarq.arquivo.arquivo\_comum.}}\sphinxbfcode{\sphinxupquote{ArquivoSimples}}}{\emph{\DUrole{n}{nome\_arquivo}\DUrole{p}{:}\DUrole{w}{  }\DUrole{n}{\sphinxhref{https://docs.python.org/3/library/stdtypes.html\#str}{str}}}, \emph{\DUrole{n}{esquema\_registro}\DUrole{p}{:}\DUrole{w}{  }\DUrole{n}{{\hyperref[\detokenize{estrutarq.registro:estrutarq.registro.registro_comum.RegistroBasico}]{\sphinxcrossref{estrutarq.registro.registro\_comum.RegistroBasico}}}}}, \emph{\DUrole{o}{**}\DUrole{n}{kwargs}}}{}
\pysigstopsignatures
\sphinxAtStartPar
Base: {\hyperref[\detokenize{estrutarq.arquivo:estrutarq.arquivo.arquivo_comum.ArquivoBasico}]{\sphinxcrossref{\sphinxcode{\sphinxupquote{estrutarq.arquivo.arquivo\_comum.ArquivoBasico}}}}}

\sphinxAtStartPar
Gerenciador de arquivo simples (como fluxo de dados) com registros de
comprimento fixo.
\index{escreva() (método estrutarq.arquivo.arquivo\_comum.ArquivoSimples)@\spxentry{escreva()}\spxextra{método estrutarq.arquivo.arquivo\_comum.ArquivoSimples}}

\begin{fulllineitems}
\phantomsection\label{\detokenize{estrutarq.arquivo:estrutarq.arquivo.arquivo_comum.ArquivoSimples.escreva}}
\pysigstartsignatures
\pysiglinewithargsret{\sphinxbfcode{\sphinxupquote{escreva}}}{\emph{\DUrole{n}{registro}\DUrole{p}{:}\DUrole{w}{  }\DUrole{n}{{\hyperref[\detokenize{estrutarq.registro:estrutarq.registro.registro_comum.RegistroBasico}]{\sphinxcrossref{estrutarq.registro.registro\_comum.RegistroBasico}}}}}, \emph{\DUrole{o}{**}\DUrole{n}{kwargs}}}{}
\pysigstopsignatures
\sphinxAtStartPar
Gravação de um registro no arquivo

\sphinxAtStartPar
self.escreva\_efetivo chama escreva\_fixo ou escreva\_variável, conforme
o registro tenha comprimento fixo ou variável

\end{fulllineitems}

\index{escreva\_fixo() (método estrutarq.arquivo.arquivo\_comum.ArquivoSimples)@\spxentry{escreva\_fixo()}\spxextra{método estrutarq.arquivo.arquivo\_comum.ArquivoSimples}}

\begin{fulllineitems}
\phantomsection\label{\detokenize{estrutarq.arquivo:estrutarq.arquivo.arquivo_comum.ArquivoSimples.escreva_fixo}}
\pysigstartsignatures
\pysiglinewithargsret{\sphinxbfcode{\sphinxupquote{escreva\_fixo}}}{\emph{\DUrole{n}{registro}\DUrole{p}{:}\DUrole{w}{  }\DUrole{n}{{\hyperref[\detokenize{estrutarq.registro:estrutarq.registro.registro_comum.RegistroBasico}]{\sphinxcrossref{estrutarq.registro.registro\_comum.RegistroBasico}}}}}, \emph{\DUrole{n}{posicao\_relativa}\DUrole{p}{:}\DUrole{w}{  }\DUrole{n}{\sphinxhref{https://docs.python.org/3/library/typing.html\#typing.Optional}{Optional}\DUrole{p}{{[}}\sphinxhref{https://docs.python.org/3/library/functions.html\#int}{int}\DUrole{p}{{]}}}\DUrole{w}{  }\DUrole{o}{=}\DUrole{w}{  }\DUrole{default_value}{None}}}{}
\pysigstopsignatures
\sphinxAtStartPar
Gravação de um registro no arquivo
:param registro: o registro a ser escrito
:param posicao\_relativa: posição relativa do registro no arquivo,
\begin{quote}

\sphinxAtStartPar
com o primeiro registro sendo o registro 0
\end{quote}

\end{fulllineitems}

\index{escreva\_variavel() (método estrutarq.arquivo.arquivo\_comum.ArquivoSimples)@\spxentry{escreva\_variavel()}\spxextra{método estrutarq.arquivo.arquivo\_comum.ArquivoSimples}}

\begin{fulllineitems}
\phantomsection\label{\detokenize{estrutarq.arquivo:estrutarq.arquivo.arquivo_comum.ArquivoSimples.escreva_variavel}}
\pysigstartsignatures
\pysiglinewithargsret{\sphinxbfcode{\sphinxupquote{escreva\_variavel}}}{\emph{\DUrole{n}{registro}\DUrole{p}{:}\DUrole{w}{  }\DUrole{n}{{\hyperref[\detokenize{estrutarq.registro:estrutarq.registro.registro_comum.RegistroBasico}]{\sphinxcrossref{estrutarq.registro.registro\_comum.RegistroBasico}}}}}, \emph{\DUrole{n}{deslocamento}\DUrole{p}{:}\DUrole{w}{  }\DUrole{n}{\sphinxhref{https://docs.python.org/3/library/typing.html\#typing.Optional}{Optional}\DUrole{p}{{[}}\sphinxhref{https://docs.python.org/3/library/functions.html\#int}{int}\DUrole{p}{{]}}}\DUrole{w}{  }\DUrole{o}{=}\DUrole{w}{  }\DUrole{default_value}{None}}}{}
\pysigstopsignatures
\sphinxAtStartPar
Gravação de um registro no arquivo
:param registro: o registro a ser escrito
:param deslocamento: posição absoluta (byte offset) da posição
\begin{quote}

\sphinxAtStartPar
de escrita
\end{quote}

\end{fulllineitems}

\index{leia() (método estrutarq.arquivo.arquivo\_comum.ArquivoSimples)@\spxentry{leia()}\spxextra{método estrutarq.arquivo.arquivo\_comum.ArquivoSimples}}

\begin{fulllineitems}
\phantomsection\label{\detokenize{estrutarq.arquivo:estrutarq.arquivo.arquivo_comum.ArquivoSimples.leia}}
\pysigstartsignatures
\pysiglinewithargsret{\sphinxbfcode{\sphinxupquote{leia}}}{\emph{\DUrole{o}{**}\DUrole{n}{kwargs}}}{{ $\rightarrow$ {\hyperref[\detokenize{estrutarq.registro:estrutarq.registro.registro_comum.RegistroBasico}]{\sphinxcrossref{estrutarq.registro.registro\_comum.RegistroBasico}}}}}
\pysigstopsignatures
\sphinxAtStartPar
Leitura de um registro do arquivo
:return: o registro lido

\sphinxAtStartPar
self.leia\_efetivo chama leia\_fixo ou leia\_variável, conforme
o registro tenha comprimento fixo ou variável

\end{fulllineitems}

\index{leia\_fixo() (método estrutarq.arquivo.arquivo\_comum.ArquivoSimples)@\spxentry{leia\_fixo()}\spxextra{método estrutarq.arquivo.arquivo\_comum.ArquivoSimples}}

\begin{fulllineitems}
\phantomsection\label{\detokenize{estrutarq.arquivo:estrutarq.arquivo.arquivo_comum.ArquivoSimples.leia_fixo}}
\pysigstartsignatures
\pysiglinewithargsret{\sphinxbfcode{\sphinxupquote{leia\_fixo}}}{\emph{\DUrole{n}{posicao\_relativa}\DUrole{p}{:}\DUrole{w}{  }\DUrole{n}{\sphinxhref{https://docs.python.org/3/library/typing.html\#typing.Optional}{Optional}\DUrole{p}{{[}}\sphinxhref{https://docs.python.org/3/library/functions.html\#int}{int}\DUrole{p}{{]}}}\DUrole{w}{  }\DUrole{o}{=}\DUrole{w}{  }\DUrole{default_value}{None}}}{{ $\rightarrow$ {\hyperref[\detokenize{estrutarq.registro:estrutarq.registro.registro_comum.RegistroBasico}]{\sphinxcrossref{estrutarq.registro.registro\_comum.RegistroBasico}}}}}
\pysigstopsignatures
\sphinxAtStartPar
Leitura de um registro de comprimento fixo
:param posicao\_relativa: posição relativa do registro no arquivo,
\begin{quote}

\sphinxAtStartPar
com o primeiro registro sendo o registro 0
\end{quote}
\begin{quote}\begin{description}
\item[{Retorna}] \leavevmode
\sphinxAtStartPar
o registro lido

\end{description}\end{quote}

\end{fulllineitems}

\index{leia\_variavel() (método estrutarq.arquivo.arquivo\_comum.ArquivoSimples)@\spxentry{leia\_variavel()}\spxextra{método estrutarq.arquivo.arquivo\_comum.ArquivoSimples}}

\begin{fulllineitems}
\phantomsection\label{\detokenize{estrutarq.arquivo:estrutarq.arquivo.arquivo_comum.ArquivoSimples.leia_variavel}}
\pysigstartsignatures
\pysiglinewithargsret{\sphinxbfcode{\sphinxupquote{leia\_variavel}}}{\emph{\DUrole{n}{posicao\_relativa}\DUrole{p}{:}\DUrole{w}{  }\DUrole{n}{\sphinxhref{https://docs.python.org/3/library/typing.html\#typing.Optional}{Optional}\DUrole{p}{{[}}\sphinxhref{https://docs.python.org/3/library/functions.html\#int}{int}\DUrole{p}{{]}}}\DUrole{w}{  }\DUrole{o}{=}\DUrole{w}{  }\DUrole{default_value}{None}}}{{ $\rightarrow$ {\hyperref[\detokenize{estrutarq.registro:estrutarq.registro.registro_comum.RegistroBasico}]{\sphinxcrossref{estrutarq.registro.registro\_comum.RegistroBasico}}}}}
\pysigstopsignatures
\sphinxAtStartPar
Leitura de um registro de comprimento variavel
:param posicao\_relativa: posição relativa do registro no arquivo,
\begin{quote}

\sphinxAtStartPar
com o primeiro registro sendo o registro 0
\end{quote}
\begin{quote}\begin{description}
\item[{Retorna}] \leavevmode
\sphinxAtStartPar
o registro lido

\end{description}\end{quote}

\sphinxAtStartPar
A determinação da posição relativa é feita por busca sequencial

\end{fulllineitems}


\end{fulllineitems}



\subsubsection{Module contents}
\label{\detokenize{estrutarq.arquivo:module-estrutarq.arquivo}}\label{\detokenize{estrutarq.arquivo:module-contents}}\index{módulo@\spxentry{módulo}!estrutarq.arquivo@\spxentry{estrutarq.arquivo}}\index{estrutarq.arquivo@\spxentry{estrutarq.arquivo}!módulo@\spxentry{módulo}}
\sphinxstepscope


\subsection{estrutarq.utilitarios package}
\label{\detokenize{estrutarq.utilitarios:estrutarq-utilitarios-package}}\label{\detokenize{estrutarq.utilitarios::doc}}

\subsubsection{Submodules}
\label{\detokenize{estrutarq.utilitarios:submodules}}

\subsubsection{estrutarq.utilitarios.disco module}
\label{\detokenize{estrutarq.utilitarios:module-estrutarq.utilitarios.disco}}\label{\detokenize{estrutarq.utilitarios:estrutarq-utilitarios-disco-module}}\index{módulo@\spxentry{módulo}!estrutarq.utilitarios.disco@\spxentry{estrutarq.utilitarios.disco}}\index{estrutarq.utilitarios.disco@\spxentry{estrutarq.utilitarios.disco}!módulo@\spxentry{módulo}}
\sphinxAtStartPar
Rotinas utilitárias gerais
\index{comprimento\_de\_bloco() (no módulo estrutarq.utilitarios.disco)@\spxentry{comprimento\_de\_bloco()}\spxextra{no módulo estrutarq.utilitarios.disco}}

\begin{fulllineitems}
\phantomsection\label{\detokenize{estrutarq.utilitarios:estrutarq.utilitarios.disco.comprimento_de_bloco}}
\pysigstartsignatures
\pysiglinewithargsret{\sphinxcode{\sphinxupquote{estrutarq.utilitarios.disco.}}\sphinxbfcode{\sphinxupquote{comprimento\_de\_bloco}}}{\emph{\DUrole{n}{diretorio}\DUrole{p}{:}\DUrole{w}{  }\DUrole{n}{\sphinxhref{https://docs.python.org/3/library/typing.html\#typing.Optional}{Optional}\DUrole{p}{{[}}\sphinxhref{https://docs.python.org/3/library/stdtypes.html\#str}{str}\DUrole{p}{{]}}}\DUrole{w}{  }\DUrole{o}{=}\DUrole{w}{  }\DUrole{default_value}{None}}}{}
\pysigstopsignatures
\sphinxAtStartPar
Determina o comprimento\_bloco de um bloco de disco, tendo como referência
o disco onde está o diretório temporário do sistema; para outro
disco, é preciso informar um diretório nesse disco em que haja direito
de criação de arquivos.
:param diretorio: um diretório no disco a ser verificado
:return: o tamanho do bloco no disco

\sphinxAtStartPar
Efeitos colaterais: é criado um arquivo temporário, que em seguida
é removido.

\end{fulllineitems}

\index{main() (no módulo estrutarq.utilitarios.disco)@\spxentry{main()}\spxextra{no módulo estrutarq.utilitarios.disco}}

\begin{fulllineitems}
\phantomsection\label{\detokenize{estrutarq.utilitarios:estrutarq.utilitarios.disco.main}}
\pysigstartsignatures
\pysiglinewithargsret{\sphinxcode{\sphinxupquote{estrutarq.utilitarios.disco.}}\sphinxbfcode{\sphinxupquote{main}}}{}{}
\pysigstopsignatures
\end{fulllineitems}



\subsubsection{estrutarq.utilitarios.dispositivo module}
\label{\detokenize{estrutarq.utilitarios:module-estrutarq.utilitarios.dispositivo}}\label{\detokenize{estrutarq.utilitarios:estrutarq-utilitarios-dispositivo-module}}\index{módulo@\spxentry{módulo}!estrutarq.utilitarios.dispositivo@\spxentry{estrutarq.utilitarios.dispositivo}}\index{estrutarq.utilitarios.dispositivo@\spxentry{estrutarq.utilitarios.dispositivo}!módulo@\spxentry{módulo}}
\sphinxAtStartPar
Rotinas utilitárias gerais
\index{comprimento\_de\_bloco() (no módulo estrutarq.utilitarios.dispositivo)@\spxentry{comprimento\_de\_bloco()}\spxextra{no módulo estrutarq.utilitarios.dispositivo}}

\begin{fulllineitems}
\phantomsection\label{\detokenize{estrutarq.utilitarios:estrutarq.utilitarios.dispositivo.comprimento_de_bloco}}
\pysigstartsignatures
\pysiglinewithargsret{\sphinxcode{\sphinxupquote{estrutarq.utilitarios.dispositivo.}}\sphinxbfcode{\sphinxupquote{comprimento\_de\_bloco}}}{\emph{\DUrole{n}{diretorio}\DUrole{p}{:}\DUrole{w}{  }\DUrole{n}{\sphinxhref{https://docs.python.org/3/library/typing.html\#typing.Optional}{Optional}\DUrole{p}{{[}}\sphinxhref{https://docs.python.org/3/library/stdtypes.html\#str}{str}\DUrole{p}{{]}}}\DUrole{w}{  }\DUrole{o}{=}\DUrole{w}{  }\DUrole{default_value}{None}}}{}
\pysigstopsignatures
\sphinxAtStartPar
Determina o comprimento\_bloco de um bloco do dispositivo externo, tendo como
referência aquele onde está o diretório temporário do sistema (parãmetro
igual a None); se um diretório em que haja direito de criação de arquivos
for informado, então o dispostivo em que ele está será utilizado.
:param diretorio: um diretório no disco a ser verificado
:return: o tamanho do bloco no disco

\sphinxAtStartPar
Efeitos colaterais: é criado um arquivo temporário, que em seguida
é removido.

\end{fulllineitems}

\index{main() (no módulo estrutarq.utilitarios.dispositivo)@\spxentry{main()}\spxextra{no módulo estrutarq.utilitarios.dispositivo}}

\begin{fulllineitems}
\phantomsection\label{\detokenize{estrutarq.utilitarios:estrutarq.utilitarios.dispositivo.main}}
\pysigstartsignatures
\pysiglinewithargsret{\sphinxcode{\sphinxupquote{estrutarq.utilitarios.dispositivo.}}\sphinxbfcode{\sphinxupquote{main}}}{}{}
\pysigstopsignatures
\end{fulllineitems}



\subsubsection{estrutarq.utilitarios.geral module}
\label{\detokenize{estrutarq.utilitarios:module-estrutarq.utilitarios.geral}}\label{\detokenize{estrutarq.utilitarios:estrutarq-utilitarios-geral-module}}\index{módulo@\spxentry{módulo}!estrutarq.utilitarios.geral@\spxentry{estrutarq.utilitarios.geral}}\index{estrutarq.utilitarios.geral@\spxentry{estrutarq.utilitarios.geral}!módulo@\spxentry{módulo}}
\sphinxAtStartPar
Funções gerais
\index{verifique\_versao() (no módulo estrutarq.utilitarios.geral)@\spxentry{verifique\_versao()}\spxextra{no módulo estrutarq.utilitarios.geral}}

\begin{fulllineitems}
\phantomsection\label{\detokenize{estrutarq.utilitarios:estrutarq.utilitarios.geral.verifique_versao}}
\pysigstartsignatures
\pysiglinewithargsret{\sphinxcode{\sphinxupquote{estrutarq.utilitarios.geral.}}\sphinxbfcode{\sphinxupquote{verifique\_versao}}}{}{}
\pysigstopsignatures
\end{fulllineitems}



\subsubsection{estrutarq.utilitarios.interpretador module}
\label{\detokenize{estrutarq.utilitarios:estrutarq-utilitarios-interpretador-module}}

\subsubsection{Module contents}
\label{\detokenize{estrutarq.utilitarios:module-estrutarq.utilitarios}}\label{\detokenize{estrutarq.utilitarios:module-contents}}\index{módulo@\spxentry{módulo}!estrutarq.utilitarios@\spxentry{estrutarq.utilitarios}}\index{estrutarq.utilitarios@\spxentry{estrutarq.utilitarios}!módulo@\spxentry{módulo}}

\renewcommand{\indexname}{Índice de Módulos Python}
\begin{sphinxtheindex}
\let\bigletter\sphinxstyleindexlettergroup
\bigletter{e}
\item\relax\sphinxstyleindexentry{estrutarq.arquivo}\sphinxstyleindexpageref{estrutarq.arquivo:\detokenize{module-estrutarq.arquivo}}
\item\relax\sphinxstyleindexentry{estrutarq.arquivo.arquivo\_comum}\sphinxstyleindexpageref{estrutarq.arquivo:\detokenize{module-estrutarq.arquivo.arquivo_comum}}
\item\relax\sphinxstyleindexentry{estrutarq.campo}\sphinxstyleindexpageref{estrutarq.campo:\detokenize{module-estrutarq.campo}}
\item\relax\sphinxstyleindexentry{estrutarq.campo.campo\_cadeia}\sphinxstyleindexpageref{estrutarq.campo:\detokenize{module-estrutarq.campo.campo_cadeia}}
\item\relax\sphinxstyleindexentry{estrutarq.campo.campo\_comum}\sphinxstyleindexpageref{estrutarq.campo:\detokenize{module-estrutarq.campo.campo_comum}}
\item\relax\sphinxstyleindexentry{estrutarq.campo.campo\_inteiro}\sphinxstyleindexpageref{estrutarq.campo:\detokenize{module-estrutarq.campo.campo_inteiro}}
\item\relax\sphinxstyleindexentry{estrutarq.campo.campo\_real}\sphinxstyleindexpageref{estrutarq.campo:\detokenize{module-estrutarq.campo.campo_real}}
\item\relax\sphinxstyleindexentry{estrutarq.campo.campo\_tempo}\sphinxstyleindexpageref{estrutarq.campo:\detokenize{module-estrutarq.campo.campo_tempo}}
\item\relax\sphinxstyleindexentry{estrutarq.dado.dado\_comum}\sphinxstyleindexpageref{estrutarq.dado:\detokenize{module-estrutarq.dado.dado_comum}}
\item\relax\sphinxstyleindexentry{estrutarq.registro.registro\_comum}\sphinxstyleindexpageref{estrutarq.registro:\detokenize{module-estrutarq.registro.registro_comum}}
\item\relax\sphinxstyleindexentry{estrutarq.utilitarios}\sphinxstyleindexpageref{estrutarq.utilitarios:\detokenize{module-estrutarq.utilitarios}}
\item\relax\sphinxstyleindexentry{estrutarq.utilitarios.disco}\sphinxstyleindexpageref{estrutarq.utilitarios:\detokenize{module-estrutarq.utilitarios.disco}}
\item\relax\sphinxstyleindexentry{estrutarq.utilitarios.dispositivo}\sphinxstyleindexpageref{estrutarq.utilitarios:\detokenize{module-estrutarq.utilitarios.dispositivo}}
\item\relax\sphinxstyleindexentry{estrutarq.utilitarios.geral}\sphinxstyleindexpageref{estrutarq.utilitarios:\detokenize{module-estrutarq.utilitarios.geral}}
\end{sphinxtheindex}

\renewcommand{\indexname}{Índice}
\printindex
\end{document}